%% LyX 2.3.7 created this file.  For more info, see http://www.lyx.org/.
%% Do not edit unless you really know what you are doing.
\documentclass[journal,article,submit,pdftex,moreauthors]{Definitions/mdpi}
\usepackage[T1]{fontenc}
\usepackage[utf8]{inputenc}
\usepackage{array}
\usepackage{float}
\usepackage{booktabs}
\usepackage{url}
\usepackage{amstext}
\usepackage{graphicx}

\makeatletter

%%%%%%%%%%%%%%%%%%%%%%%%%%%%%% LyX specific LaTeX commands.

\Title{Optimization: parallel stochastic methods and mixed termination rules}

\TitleCitation{Optimization: parallel stochastic methods and mixed termination rules}

\Author{Vasileios Charilogis$^{1}$, Ioannis G. Tsoulos$^{2,*}$, Anna Maria
Gianni$^{3}$}

\AuthorNames{V. Charilogis, I.G. Tsoulos, A.M. Gianni}

\AuthorCitation{Charilogis V.; Tsoulos I.G.; Gianni A.M; }


\address{$^{1}$\quad{}Department of Informatics and Telecommunications,
University of Ioannina, Greece; v.charilog@uoi.gr\\
$^{2}$\quad{}Department of Informatics and Telecommunications, University
of Ioannina, Greece; itsoulos@uoi.gr\\
$^{3}$\quad{}Department of Informatics and Telecommunications, University
of Ioannina, Greece; am.gianni@uoi.gr}


\corres{Correspondence: itsoulos@uoi.gr;}


\firstnote{Current address: Department of Informatics and Telecommunications,
University of Ioannina, Greece.}


\secondnote{These authors contributed equally to this work.}


\abstract{Parallel optimization constitutes a powerful tool for solving optimization
problems in various domains. By harnessing multiple computational
resources simultaneously, optimization methods can achieve faster
convergence and improved performance. This specific parallel implementation
involves the concurrent execution of different methods, such as evolutionary
algorithms, swarm-based optimization, and the utilization of multiple
restarts. The primary objective is the efficient exploration of the
search space and the attainment of optimal solutions in shorter time
frames without squandering computational power. However, defined termination
criteria are essential to prevent uncontrolled execution of the algorithm,
aiming to conserve computational resources and time. Within the scope
of this study, an innovative combination of termination rules and
a mechanism for transferring optimal solutions among different methodological
approaches is proposed. The proposed enhancements have been tested
on a series of well-known optimization problems from relevant literature,
and the results are reported here.}


\keyword{Global optimization; Parallel techniques; Termination rules; Evolutionary
techniques}

%% Because html converters don't know tabularnewline
\providecommand{\tabularnewline}{\\}
\floatstyle{ruled}
\newfloat{algorithm}{tbp}{loa}
\providecommand{\algorithmname}{Algorithm}
\floatname{algorithm}{\protect\algorithmname}

%%%%%%%%%%%%%%%%%%%%%%%%%%%%%% User specified LaTeX commands.
%  LaTeX support: latex@mdpi.com 
%  For support, please attach all files needed for compiling as well as the log file, and specify your operating system, LaTeX version, and LaTeX editor.

%=================================================================


% For posting an early version of this manuscript as a preprint, you may use "preprints" as the journal and change "submit" to "accept". The document class line would be, e.g., \documentclass[preprints,article,accept,moreauthors,pdftex]{mdpi}. This is especially recommended for submission to arXiv, where line numbers should be removed before posting. For preprints.org, the editorial staff will make this change immediately prior to posting.

%--------------------
% Class Options:
%--------------------
%----------
% journal
%----------
% Choose between the following MDPI journals:
% acoustics, actuators, addictions, admsci, adolescents, aerospace, agriculture, agriengineering, agronomy, ai, algorithms, allergies, alloys, analytica, animals, antibiotics, antibodies, antioxidants, applbiosci, appliedchem, appliedmath, applmech, applmicrobiol, applnano, applsci, aquacj, architecture, arts, asc, asi, astronomy, atmosphere, atoms, audiolres, automation, axioms, bacteria, batteries, bdcc, behavsci, beverages, biochem, bioengineering, biologics, biology, biomass, biomechanics, biomed, biomedicines, biomedinformatics, biomimetics, biomolecules, biophysica, biosensors, biotech, birds, bloods, blsf, brainsci, breath, buildings, businesses, cancers, carbon, cardiogenetics, catalysts, cells, ceramics, challenges, chemengineering, chemistry, chemosensors, chemproc, children, chips, cimb, civileng, cleantechnol, climate, clinpract, clockssleep, cmd, coasts, coatings, colloids, colorants, commodities, compounds, computation, computers, condensedmatter, conservation, constrmater, cosmetics, covid, crops, cryptography, crystals, csmf, ctn, curroncol, currophthalmol, cyber, dairy, data, dentistry, dermato, dermatopathology, designs, diabetology, diagnostics, dietetics, digital, disabilities, diseases, diversity, dna, drones, dynamics, earth, ebj, ecologies, econometrics, economies, education, ejihpe, electricity, electrochem, electronicmat, electronics, encyclopedia, endocrines, energies, eng, engproc, ent, entomology, entropy, environments, environsciproc, epidemiologia, epigenomes, est, fermentation, fibers, fintech, fire, fishes, fluids, foods, forecasting, forensicsci, forests, foundations, fractalfract, fuels, futureinternet, futureparasites, futurepharmacol, futurephys, futuretransp, galaxies, games, gases, gastroent, gastrointestdisord, gels, genealogy, genes, geographies, geohazards, geomatics, geosciences, geotechnics, geriatrics, hazardousmatters, healthcare, hearts, hemato, heritage, highthroughput, histories, horticulturae, humanities, humans, hydrobiology, hydrogen, hydrology, hygiene, idr, ijerph, ijfs, ijgi, ijms, ijns, ijtm, ijtpp, immuno, informatics, information, infrastructures, inorganics, insects, instruments, inventions, iot, j, jal, jcdd, jcm, jcp, jcs, jdb, jeta, jfb, jfmk, jimaging, jintelligence, jlpea, jmmp, jmp, jmse, jne, jnt, jof, joitmc, jor, journalmedia, jox, jpm, jrfm, jsan, jtaer, jzbg, kidney, kidneydial, knowledge, land, languages, laws, life, liquids, literature, livers, logics, logistics, lubricants, lymphatics, machines, macromol, magnetism, magnetochemistry, make, marinedrugs, materials, materproc, mathematics, mca, measurements, medicina, medicines, medsci, membranes, merits, metabolites, metals, meteorology, methane, metrology, micro, microarrays, microbiolres, micromachines, microorganisms, microplastics, minerals, mining, modelling, molbank, molecules, mps, msf, mti, muscles, nanoenergyadv, nanomanufacturing, nanomaterials, ncrna, network, neuroglia, neurolint, neurosci, nitrogen, notspecified, nri, nursrep, nutraceuticals, nutrients, obesities, oceans, ohbm, onco, oncopathology, optics, oral, organics, organoids, osteology, oxygen, parasites, parasitologia, particles, pathogens, pathophysiology, pediatrrep, pharmaceuticals, pharmaceutics, pharmacoepidemiology, pharmacy, philosophies, photochem, photonics, phycology, physchem, physics, physiologia, plants, plasma, pollutants, polymers, polysaccharides, poultry, powders, preprints, proceedings, processes, prosthesis, proteomes, psf, psych, psychiatryint, psychoactives, publications, quantumrep, quaternary, qubs, radiation, reactions, recycling, regeneration, religions, remotesensing, reports, reprodmed, resources, rheumato, risks, robotics, ruminants, safety, sci, scipharm, seeds, sensors, separations, sexes, signals, sinusitis, skins, smartcities, sna, societies, socsci, software, soilsystems, solar, solids, sports, standards, stats, stresses, surfaces, surgeries, suschem, sustainability, symmetry, synbio, systems, taxonomy, technologies, telecom, test, textiles, thalassrep, thermo, tomography, tourismhosp, toxics, toxins, transplantology, transportation, traumacare, traumas, tropicalmed, universe, urbansci, uro, vaccines, vehicles, venereology, vetsci, vibration, viruses, vision, waste, water, wem, wevj, wind, women, world, youth, zoonoticdis 

%---------
% article
%---------
% The default type of manuscript is "article", but can be replaced by: 
% abstract, addendum, article, book, bookreview, briefreport, casereport, comment, commentary, communication, conferenceproceedings, correction, conferencereport, entry, expressionofconcern, extendedabstract, datadescriptor, editorial, essay, erratum, hypothesis, interestingimage, obituary, opinion, projectreport, reply, retraction, review, perspective, protocol, shortnote, studyprotocol, systematicreview, supfile, technicalnote, viewpoint, guidelines, registeredreport, tutorial
% supfile = supplementary materials

%----------
% submit
%----------
% The class option "submit" will be changed to "accept" by the Editorial Office when the paper is accepted. This will only make changes to the frontpage (e.g., the logo of the journal will get visible), the headings, and the copyright information. Also, line numbering will be removed. Journal info and pagination for accepted papers will also be assigned by the Editorial Office.

%------------------
% moreauthors
%------------------
% If there is only one author the class option oneauthor should be used. Otherwise use the class option moreauthors.

%---------
% pdftex
%---------
% The option pdftex is for use with pdfLaTeX. If eps figures are used, remove the option pdftex and use LaTeX and dvi2pdf.

%=================================================================
% MDPI internal commands - do not modify
\firstpage{1} 
 
\setcounter{page}{\@firstpage} 

\pubvolume{1}
\issuenum{1}
\articlenumber{0}
\pubyear{2023}
\copyrightyear{2023}
%\externaleditor{Academic Editor: Firstname Lastname} % For journal Automation, please change Academic Editor to "Communicated by"
\datereceived{}
\daterevised{ } % Comment out if no revised date
\dateaccepted{}
\datepublished{}
%\datecorrected{} % Corrected papers include a "Corrected: XXX" date in the original paper.
%\dateretracted{} % Corrected papers include a "Retracted: XXX" date in the original paper.
\hreflink{https://doi.org/} % If needed use \linebreak
%\doinum{}
%------------------------------------------------------------------
% The following line should be uncommented if the LaTeX file is uploaded to arXiv.org
%\pdfoutput=1

%=================================================================
% Add packages and commands here. The following packages are loaded in our class file: fontenc, inputenc, calc, indentfirst, fancyhdr, graphicx, epstopdf, lastpage, ifthen, lineno, float, amsmath, setspace, enumitem, mathpazo, booktabs, titlesec, etoolbox, tabto, xcolor, soul, multirow, microtype, tikz, totcount, changepage, attrib, upgreek, cleveref, amsthm, hyphenat, natbib, hyperref, footmisc, url, geometry, newfloat, caption

%=================================================================
%% Please use the following mathematics environments: Theorem, Lemma, Corollary, Proposition, Characterization, Property, Problem, Example, ExamplesandDefinitions, Hypothesis, Remark, Definition, Notation, Assumption
%% For proofs, please use the proof environment (the amsthm package is loaded by the MDPI class).

%=================================================================
% The fields PACS, MSC, and JEL may be left empty or commented out if not applicable
%\PACS{J0101}
%\MSC{}
%\JEL{}

%%%%%%%%%%%%%%%%%%%%%%%%%%%%%%%%%%%%%%%%%%
% Only for the journal Diversity
%\LSID{\url{http://}}

%%%%%%%%%%%%%%%%%%%%%%%%%%%%%%%%%%%%%%%%%%
% Only for the journal Applied Sciences:
%\featuredapplication{Authors are encouraged to provide a concise description of the specific application or a potential application of the work. This section is not mandatory.}
%%%%%%%%%%%%%%%%%%%%%%%%%%%%%%%%%%%%%%%%%%

%%%%%%%%%%%%%%%%%%%%%%%%%%%%%%%%%%%%%%%%%%
% Only for the journal Data:
%\dataset{DOI number or link to the deposited data set in cases where the data set is published or set to be published separately. If the data set is submitted and will be published as a supplement to this paper in the journal Data, this field will be filled by the editors of the journal. In this case, please make sure to submit the data set as a supplement when entering your manuscript into our manuscript editorial system.}

%\datasetlicense{license under which the data set is made available (CC0, CC-BY, CC-BY-SA, CC-BY-NC, etc.)}

%%%%%%%%%%%%%%%%%%%%%%%%%%%%%%%%%%%%%%%%%%
% Only for the journal Toxins
%\keycontribution{The breakthroughs or highlights of the manuscript. Authors can write one or two sentences to describe the most important part of the paper.}

%%%%%%%%%%%%%%%%%%%%%%%%%%%%%%%%%%%%%%%%%%
% Only for the journal Encyclopedia
%\encyclopediadef{Instead of the abstract}
%\entrylink{The Link to this entry published on the encyclopedia platform.}
%%%%%%%%%%%%%%%%%%%%%%%%%%%%%%%%%%%%%%%%%%

%%%%%%%%%%%%%%%%%%%%%%%%%%%%%%%%%%%%%%%%%%
% Only for the journal Advances in Respiratory Medicine
%\addhighlights{yes}
%\renewcommand{\addhighlights}{%

%\noindent This is an obligatory section in “Advances in Respiratory Medicine”, whose goal is to increase the discoverability and readability of the article via search engines and other scholars. Highlights should not be a copy of the abstract, but a simple text allowing the reader to quickly and simplified find out what the article is about and what can be cited from it. Each of these parts should be devoted up to 2~bullet points.\vspace{3pt}\\
%\textbf{What are the main findings?}
% \begin{itemize}[labelsep=2.5mm,topsep=-3pt]
% \item First bullet.
% \item Second bullet.
% \end{itemize}\vspace{3pt}
%\textbf{What is the implication of the main finding?}
% \begin{itemize}[labelsep=2.5mm,topsep=-3pt]
% \item First bullet.
% \item Second bullet.
% \end{itemize}
%}
%%%%%%%%%%%%%%%%%%%%%%%%%%%%%%%%%%%%%%%%%%

\makeatother

\begin{document}
\maketitle

\section{Introduction}

The global minimization problem of a continuous and differentiable
function $f:S\rightarrow R,S\subset R^{n}$ can be formulated as
\begin{equation}
x^{*}=\mbox{arg}\min_{x\in S}f(x).\label{eq:eq1-1}
\end{equation}
where the set $S$ is has as follows: \textbf{
\[
S=\left[a_{1},b_{1}\right]\otimes\left[a_{2},b_{2}\right]\otimes\ldots\left[a_{n},b_{n}\right]
\]
}Such problems occur frequently in areas such as physics \citep{go_physics1,go_physics2,go_physics3},
chemistry \citep{go_chemistry1,go_chemistry2,go_chemistry3}, economics
\citep{go_econ1,go_econ2}, medicine \citep{go_med1,go_med2} etc.
A series of methods have been proposed in the recent literature to
handle problems of Equation \ref{eq:eq1-1}. Methods used to handle
problems of Equation \ref{eq:eq1-1} are usually divided into two
categories:\textbf{ }deterministic and stochastic methods. In the
first category, the most common method is the interval method \citep{interval1,interval2},
where the set $S$ is divided through a series of steps into subregions
and some subregions that do not contain the global solution can be
removed using some pre-defined criteria. On the other hand, there
is a variety of stochastic methods that are easier to implement than
the deterministic ones. Among them, there are methods based on considerations
derived from Physics, such as the Simulated Annealing method\citep{Kirkpatrick},
the Henry's Gas Solubility Optimization (HGSO) \citep{Hashim}, the
Gravitational Search Algorithm (GSA) \citep{Rashedi}, the Small World
Optimization Algorithm (SWOA) \citep{Du-Wu}, etc. Also, a series
of evolutionary based techniques have also been suggested for Global
Optimization problems, such as Genetic Algorithms \citep{Goldberg},
the Differential Evolution method \citep{Das,Charilogis}, Particle
Swarm Optimization (PSO) \citep{Kennedy,Charilogis2}, Ant Colony
Optimization (ACO) \citep{Dorigo}, Bat Algorithm (BA) \citep{Yang},
Whale Optimization Algorithm (WOA) \citep{Mirjalili}, Grasshopper
Optimization Algorithm (GOA) \citep{Saremi}, etc.

However, the above optimization methods require significant amounts
of computational power and time; therefore, parallelization of the
methods is essential. Recently, a variety of methods proposed that
can take advantage of parallel processing, such as parallel techniques
\citep{parallel-pso,parallel-multistart,parallel-doublepop}, or methods
that utilize the GPU architectures \citep{msgpu1,msgpu2,msgpu3} etc.\textbf{
}Parallel optimization represents a significant approach in the field
of optimization problem-solving and is applied across a wide range
of applications, such as optimization of machine learning model parameters
\citep{Low Gon,Yangyang,Yamashiro}, control system design and optimization
\citep{Kim Tsai,Oh Jong,Fatehi}, energy and resource management \citep{Cai Yang,Yu Shahabi,Ramirez,Tavakolan},
as well as problems related to sustainable development and enhancing
sustainability \citep{Lin,Pang,Ezugwu}.

By harnessing multiple computational resources, parallel optimization
allows for the simultaneous execution of multiple algorithms, leading
to faster convergence and improved performance. These computational
resources can communicate with each other to exchange information
and synchronize processes, thereby contributing to faster convergence
towards common solutions. Additionally, leveraging multiple resources
enables more effective handling of exceptions and errors, while increased
computational power for conducting more trials or utilizing more complex
models leads to enhanced performance \citep{Censor}. Of course, this
process requires the development of suitable algorithms and techniques
to effectively manage and exploit available resources. Each parallel
optimization algorithm requires a coherent strategy for workload distribution
among the available resources, as well as an efficient method for
collecting and evaluating results.

Parallel techniques have been developed by various researchers, such
as\textbf{ }combining Simulated Annealing and parallel techniques
\citep{go_par1}, parallel Particle Swarm Optimization methods \citep{go_par2},\textbf{
}application of radial basis functions in parallel stochastic optimization
\citep{go_par3} etc. Also, since the Genetic Algorithms is a method
that can be parallelized easily, there are several researchers that
have examined them thoroughly in the relevant literature \citep{pga_survey1,pga_survey2}.

In this paper, a new optimization method is proposed which is a mixture
of existing global optimization techniques running in parallel on
a number of available computing units. Each technique is executed
independently of the others and periodically the optimal values retrieved
from them are distributed to the rest of the computing units using
the propagation techniques presented here. In addition, for the most
efficient termination of the overall algorithm, intelligent termination
techniques based on stochastic observations are used, which are suitably
modified to adapt to the parallel computing environment. The proposed
algorithm is graphically outlined in Figure \ref{fig:basicFlowchart}.
The methods that are used in each processing unit may include Particle
Swarm Optimization (PSO) , the Multistart method and Differential
Evolution (DE). These techniques have been adopted since they have
been widely used in the relevant literature and provide the possibility
to be parallelized relatively easily.

The following sections are organized as follows: In section \ref{sec:Methods-description},
the three main algorithms participating in the overall algorithm are
described. In section \ref{sec:The-proposed-method} the algorithm
for parallelizing the three methods is described, along with the proposed
mechanism for propagating the optimal solution to the remaining methods.
In section \ref{sec:Experiments}, experimental models and experimental
results are described. Finally, in section \ref{sec:Conclusions},
the conclusions from the application of the current work are discussed.

\begin{figure}[H]
\begin{centering}
\includegraphics[scale=0.3]{pm_datagram}\caption{Flowchart of the overall process\label{fig:basicFlowchart}. The DE
acronym stands for the Differential Evolution method, the PSO acroym
stands for the Particle Swarm Optimization method and the PU acronym
represents the Parallel Unit that executes the algorithm. }
\par\end{centering}
\end{figure}


\section{Adopted Algorithms\label{sec:Methods-description}}

The methods that may executed in each processing unit are fully described
in this section.

\subsection{The method Particle swarm optimization\label{subsec:Particle_swarm_optimization}}

PSO is an optimization method inspired by the behavior of swarms in
nature. In PSO, a set of particles moves in the search space seeking
the optimal solution. Each particle has a current position and velocity,
and it moves based on its past performance and that of its neighboring
particles. PSO continuously adjusts the movement of particles aiming
for convergence to the optimal solution \citep{Marini,Garc=0000EDa-Gonzalo,Jain}.
This method can be programmed easily and the set of the parameters
to be set is limited. Hence, it has been used in a series of practical
problems, such as problems that arise in physics \citep{psophysics1,psophysics2},
chemistry \citep{psochem1,psochem2}, medicine \citep{psomed1,psomed2},
economics \citep{psoecon} etc. The main steps of the PSO method are
presented in the Algorithm \ref{alg:PSO}.

\begin{algorithm}[H]
\caption{The main steps of the PSO method.\label{alg:PSO}}

\begin{enumerate}
\item \textbf{Initialization}. 

\begin{enumerate}
\item \textbf{Set} $\text{\mbox{iter}}=0$ (iteration counter).
\item \textbf{Set} the number of particles \textbf{$N_{p}$}.
\item \textbf{Set} the maximum number of iterations allowed $\mbox{iter}_{\mbox{max}}$
\item \textbf{Set} the local search rate $p_{l}\in[0,1]$.
\item \textbf{Initialize} randomly the positions of the $m$ particles $x_{1},x_{2},...,x_{m}$,
with $x_{i}\in S\subset R^{n}$
\item \textbf{Initialize} randomly the velocities of the $m$ particles
$u_{1},u_{2},...,u_{m}$, with $u_{i}\in S\subset R^{n}$
\item \textbf{For} $i=1..N_{p}$ do $p_{i}=x_{i}$. The $p_{i}$ vector
are the best located values for every particle $i$.
\item \textbf{Set} $p_{\mbox{best}}=\arg\min_{i\in1..m}f\left(x_{i}\right)$
\end{enumerate}
\item \textbf{Termination Check}.\label{enu:TCheckPSO} Check for termination.
If termination criteria are met then stop.
\item \textbf{For} $i=1..N_{p}$ \textbf{Do\label{enu:For}}

\begin{enumerate}
\item \textbf{Update} the velocity:

\[
u_{i}=\omega u_{i}+r_{1}c_{1}\left(p_{i}-x_{i}\right)+r_{2}c_{2}\left(p_{\mbox{best}}-x_{i}\right)
\]
The parameters $r_{1},\ r_{2}$ are random numbers with $r_{1}\in[0,1]$
and $r_{2}\in[0,1]$.

The constant number $c_{1},\ c_{2}$ are in the range $[1,2]$.

The variable $\omega$ is called inertia, with $\omega\in[0,1]$.
\item \textbf{Update} the position 
\[
x_{i}=x_{i}+u_{i}
\]
\item \textbf{Set} $r\in[0,1]$ a random number. If $r\le p_{m}$ then $x_{i}=\mbox{LS}\left(x_{i}\right)$,
where $\mbox{LS}(x)$ is a local search procedure.
\item \textbf{Evaluate} the fitness of the particle $i$, $f\left(x_{i}\right)$
\item \textbf{If} $f\left(x_{i}\right)\le f\left(p_{i}\right)$ then $p_{i}=x_{i}$
\end{enumerate}
\item \textbf{End} \textbf{For}
\item \textbf{Set} $p_{\mbox{best}}=\arg\min_{i\in1..m}f\left(x_{i}\right)$
\item \textbf{Set} $\mbox{iter}=\mbox{iter}+1$.
\item \textbf{Goto} Step \ref{enu:TCheckPSO}
\end{enumerate}
\end{algorithm}


\subsection{The method Differential evolution\label{subsec:Differential_evolution}}

The DE method relies on differential operators and is particularly
effective in optimization problems that involve searching through
a continuous search space. By employing differential operators, DE
generates new solutions that are then evaluated and adjusted until
the optimal solution is achieved \citep{Feoktistov,Bilal}. The method
was used in a series of problems, such as electromagnetics \citep{de_app1},
energy consumption problems \citep{de_app2}, job shop scheduling
\citep{de_app3}, image segmentation \citep{de_app4} etc.\textbf{
}The Differential Evolution operates through the following steps:
Initially, initialization takes place with a random population of
solutions in the search space. Then, each solution is evaluated based
on the objective function. Subsequently, a process is iterated involving
the generation of new solutions through modifications, evaluation
of these new solutions, and selection of the best ones for the next
generation. The algorithm terminates when a termination criterion
is met, such as achieving sufficient improvement in performance or
exhausting the number of iterations. The main steps of the DE method
are presented in Algorithm \ref{alg:DE}.

\begin{algorithm}[H]
\caption{The main steps of the DE method.\label{alg:DE}}

\begin{enumerate}
\item \textbf{INPUT:}
\begin{enumerate}
\item The population size $\mbox{\ensuremath{N_{d}}}\ge$4. The members
of this population are also called agents.
\item The crossover probability $\mbox{CR}\in[0,1]$.
\item The differential weight $F\in[0,2].$
\end{enumerate}
\item \textbf{OUTPUT}:
\begin{enumerate}
\item The agent $x_{\mbox{best}}$ with the lower function value $f\left(x_{\mbox{best}}\right)$.
\end{enumerate}
\item \textbf{Initialize} all agents in $S$.
\item \textbf{While} termination criteria are not hold \textbf{do}
\begin{enumerate}
\item \textbf{For} $i=1\ldots\mbox{\ensuremath{N_{d}}}\ $\textbf{do}
\begin{enumerate}
\item \textbf{Select} as $x$ the agent $i$.
\item \textbf{Select} randomly three agents $a,b,c$ with the property $a\ne b,\ b\ne c,\ c\ne a$.
\item \textbf{Select} a random position $R\in\left\{ 1,\ldots,n\right\} $
\item \textbf{Create} the vector $y=\left[y_{1,}y_{2},\ldots,y_{n}\right]$
with the following procedure
\item \textbf{For} $j=1,\ldots,n$ \textbf{do}
\begin{enumerate}
\item \textbf{Set} $r_{i}\in[0,1]$ a random number.
\item \textbf{If} $r_{j}<\text{\mbox{CR} }$\textbf{or} $j=R$ \textbf{then}
$y_{j}=a_{j}+F\times\left(b_{j}-c_{j}\right)$ \textbf{else} $y_{j}=x_{j}$.
\end{enumerate}
\item If $y\in S\ \mbox{AND}\ f\left(y\right)\le f\left(x\right)$ then
$x=y$.
\item \textbf{EndFor}
\end{enumerate}
\item \textbf{EndFor}
\end{enumerate}
\item \textbf{End While}
\end{enumerate}
\end{algorithm}


\subsection{The method Multistart\label{subsec:Multistart}}

In contrast, the Multistart method follows a different approach than
previous methods. Instead of focusing on a single initial point, it
repeats the search from various initial points in the search space.
This allows for a broader exploration of the search space and increases
the chances of finding the optimal solution\textbf{.} The Multistart
method is particularly useful when optimization algorithms may get
trapped in local minima \citep{Marti,Marti Moreno,Tu Mayne}. The
approach of multiple starts belongs to the simplest techniques for
global optimization. At the outset of the process, an initial distribution
of points is made in Euclidean space. Subsequently, local optimization
begins simultaneously from these points. The discovered minima are
compared, and the best one is retained as the global minimum. Local
optimizations rely on the Broyden Fletcher Goldfarb Shanno (BFGS)
method \citep{bfgs_main}. The main steps of the multistart method
are presented in Algorithm \ref{alg:multistart}.

\begin{algorithm}[H]
\caption{The main steps of the Multistart method.\label{alg:multistart}}

\begin{enumerate}
\item \textbf{Initialization} step.
\begin{enumerate}
\item \textbf{Set} \textbf{$N_{m}$ }as the total number of samples.
\item \textbf{Set} $\left(x^{*},y^{*}\right)$ as the global minimum. Initialize
$y^{*}$ to a very large value.
\end{enumerate}
\item \textbf{Sampling} step.
\begin{enumerate}
\item \textbf{For} $i=1\ldots N_{m}$ \textbf{Do}
\begin{enumerate}
\item \textbf{Sample} a point $x_{i}\in S$
\item $y_{i}=\mbox{LS}\left(x_{i}\right)$. Where LS(x) is a local search
procedure.
\item \textbf{If} $y_{i}\le y^{*}$ then $x^{*}=x_{i},y^{*}=y_{i}$
\end{enumerate}
\item \textbf{EndFor}
\end{enumerate}
\end{enumerate}
\end{algorithm}


\section{The proposed method\label{sec:The-proposed-method}}

In the present work, a mixture of termination rules is proposed as
a novel technique, which can be used without partitioning in any stochastic
global optimization technique. Also, a new parallel global optimization
technique is proposed that utilizes the new termination rule. In the
following subsections, the new termination rule is fully described,
followed by the new optimization method.

\subsection{The new termination rule}

The proposed termination rule is a mixture of several stochastic termination
rules proposed in the recent literature. The first algorithm will
be called \emph{best - fitness }and in this termination technique,
at each iteration \emph{$k$}, the difference between the current
best value $f_{min}^{(k)}$ and the previous best value $f_{min}^{(k+1)}$
is calculated, i.e., the absolute difference:
\begin{center}
\begin{equation}
\left|f_{\mbox{min}}^{(k)}-f_{\mbox{min}}^{(k-1)}\right|\label{eq:best}
\end{equation}
\par\end{center}

If this difference is zero for a series of predefined consecutive
iterations $N_{k}$, then the method terminates. In the second termination
rule, which will be called \emph{mean - fitness}, the average function
value for each iteration is computed. If this value remains relatively
unchanged over a certain number of consecutive iterations, it indicates
that the method may not be making significant progress towards discovering
a new global minimum. Therefore, termination is warranted. Hence,
in every iteration $k$, we compute:

\begin{equation}
\text{\ensuremath{\delta^{(k)}=}}\left|\sum_{i=1}^{\mbox{NP}}\left|f_{i}^{(k)}\right|-\sum_{i=1}^{\mbox{NP}}\left|f_{i}^{(k-1)}\right|\right|\label{eq:mean}
\end{equation}

The value NP denotes the number of particles or chromosomes that participate
in the algorithm. The termination rule is defined as follows: terminate
if $\delta^{(k)}\le\epsilon$ for a predefined number $N_{k}$ of
iterations. The third stopping rule was the so - called DoubleBox
stopping rule, that was initially proposed in the work of Tsoulos
\citep{Tsoulos}. According to this criterion, the algorithm terminates
when one of the following conditions is met:
\begin{itemize}
\item The iteration $k$ count exceeds a predefined limit of iterations
$k_{max}$
\item The relative variance $\sigma^{(k)}$ falls below half of the variance
$\sigma^{(k_{last})}$ of the last iteration $k_{(last)}$ where a
new optimal functional value was found.
\begin{equation}
k\geq N_{k}\,or\,\sigma^{(k)}\leq\frac{\sigma^{(klast)}}{2}\label{eq:double}
\end{equation}
\\
This proposed termination criterion is the combination of the aforementioned
similarity types. Optimization termination is achieved when any homogeneity
condition is satisfied, resulting in improved speed. Therefore, the
following relationships holds:
\end{itemize}
\[
\left|f_{\mbox{min}}^{(k)}-f_{\mbox{min}}^{(k-1)}\right|or
\]

\[
\text{\ensuremath{\delta^{(k)}=}}\left|\sum_{i=1}^{\mbox{NP}}\left|f_{i}^{(k)}\right|-\sum_{i=1}^{\mbox{NP}}\left|f_{i}^{(k-1)}\right|\right|or
\]

\[
\sigma^{(iteration)}\leq\frac{\sigma^{(klast)}}{2}\,or
\]

\[
k\geq N_{k}
\]


\subsection{The proposed algorithm }

In the scientific literature, parallelization typically involves distributing
the population among parallel computing units to save computational
power and time \citep{pde,ppso}. In the present study, we concurrently
compute the optimal solutions of three different stochastic optimization
methods originating from different classification categories. The
proposed algorithm is shown in Algorithm \ref{alg:The-overall-algorithm}.

\begin{algorithm}[H]
\begin{enumerate}
\item \textbf{Set} as $N_{I}$ the total number of parallel processing units.
\item \textbf{Set} as $N_{k}$ the total number of allowed iterations.
\item \textbf{Set} $k=0$ the iteration number.
\item \textbf{For} $j=1,..,N_{I}$ do in parallel\label{enu:For--do}
\begin{enumerate}
\item \textbf{Execute} an iteration of any stochastic algorithm mentioned
in section \ref{sec:Methods-description}.
\item \textbf{Find} the best element from all optimization methods and \textbf{propagate}
it to the rest of processing units.
\end{enumerate}
\item \textbf{End For}
\item \textbf{Update} $k=k+1$
\item \textbf{Check} the proposed termination rule. If the termination rule
is valid, then goto step \ref{enu:Terminate-and-report} else goto
step \ref{enu:For--do}.
\begin{enumerate}
\item \textbf{Terminate} and report the best value from all processing units.
\label{enu:Terminate-and-report} 
\end{enumerate}
\end{enumerate}
\caption{The proposed overall algorithm\label{alg:The-overall-algorithm}}
\end{algorithm}
 

\section{Experiments\label{sec:Experiments}}

All experiments conducted were repeated 30 times to ensure the reliability
of the algorithm producing the results. The parallelization was achieved
using the OpenMP library \citep{openMP}, while the implementation
of the method was done in ANSI C++ within the optimization package
OPTIMUS, available at \url{https://github.com/itsoulos/OPTIMUS}.
The values used for all the parameters are shown in Table \ref{tab:settings}.
The values of the parameters used in the experiments were such that
there is a balance between the ability of the algorithms to discover
the global minimum with certainty on the one hand, but also on their
speed.

\begin{table}[H]
\centering{}\caption{The following parameters were considered for conducting the experiments\label{tab:settings}}
\begin{tabular}{>{\centering}p{3cm}>{\centering}p{5.8cm}c}
\toprule 
Parameter & Value & Explanation\tabularnewline
\midrule
\midrule 
$N_{e}$ & 120 & Total elements\tabularnewline
\midrule 
$N_{k}$ & 200 & Maximum number of iterations\tabularnewline
\midrule 
$N_{s}$ & 15 (DE, PSO and Multistart) & Similarity max count\tabularnewline
\midrule 
$F$ & 0.8 (DE) & Differential weight for DE\tabularnewline
\midrule 
$CR$ & 0.9 (DE) & Crossover Probability for DE\tabularnewline
\bottomrule
\end{tabular}
\end{table}


\subsection{Test functions}

The test functions \citep{Floudas,Montaz Ali} presented below exhibit
varying levels of difficulty in solving them; hence, a periodic local
optimization mechanism has been incorporated. Periodic local optimization
plays a crucial role in increasing the success rate in locating the
minimum of functions. This addition appears to lead to a success rate
approaching 100\% for all functions, regardless of their characteristics
such as dimensionality, minima, scalability, and symmetry. A study
by Z.-M. Gao and colleagues \citep{Gao} specifically examines the
issue of symmetry and asymmetry in the test functions. The test functions
used in the conducted experiments are shown in Table \ref{tab:The-test-functions}.
The application of a local search method at the end of every method
used in the experiments ensures that the global optimization method
will discover a true minima of the used test function.

\begin{table}[H]
\begin{centering}
\begin{tabular}{|c|c|c|}
\hline 
{\footnotesize{}NAME} & {\scriptsize{}FORMULA} & {\scriptsize{}DIMENSION}\tabularnewline
\hline 
\hline 
{\footnotesize{}Bent Cigar} & {\scriptsize{}$f(x)=x_{1}^{2}+10^{6}\sum_{i=2}^{n}x_{i}^{2}$} & {\scriptsize{}10}\tabularnewline
\hline 
{\footnotesize{}BF1} & {\scriptsize{}$f(x)=x_{1}^{2}+2x_{2}^{2}-\frac{3}{10}\cos\left(3\pi x_{1}\right)-\frac{4}{10}\cos\left(4\pi x_{2}\right)+\frac{7}{10}$} & {\scriptsize{}2}\tabularnewline
\hline 
{\footnotesize{}BF2} & {\scriptsize{}$f(x)=x_{1}^{2}+2x_{2}^{2}-\frac{3}{10}\cos\left(3\pi x_{1}\right)\cos\left(4\pi x_{2}\right)+\frac{3}{10}$} & {\scriptsize{}2}\tabularnewline
\hline 
{\footnotesize{}Branin} & {\scriptsize{}$f(x)=\left(x_{2}-\frac{5.1}{4\pi^{2}}x_{1}^{2}+\frac{5}{\pi}x_{1}-6\right)^{2}+10\left(1-\frac{1}{8\pi}\right)\cos(x_{1})+10$ } & {\scriptsize{}2}\tabularnewline
\hline 
{\footnotesize{}CM} & {\scriptsize{}$f(x)=\sum_{i=1}^{n}x_{i}^{2}-\frac{1}{10}\sum_{i=1}^{n}\cos\left(5\pi x_{i}\right)$} & {\scriptsize{}4}\tabularnewline
\hline 
{\footnotesize{}Discus} & {\scriptsize{}$f(x)=10^{6}x_{1}^{2}+\sum_{i=2}^{n}x_{i}^{2}$} & {\scriptsize{}10}\tabularnewline
\hline 
{\footnotesize{}Easom} & {\scriptsize{}$f(x)=-\cos\left(x_{1}\right)\cos\left(x_{2}\right)\exp\left(\left(x_{2}-\pi\right)^{2}-\left(x_{1}-\pi\right)^{2}\right)$} & {\scriptsize{}2}\tabularnewline
\hline 
{\footnotesize{}Exp} & {\scriptsize{}$f(x)=-\exp\left(-0.5\sum_{i=1}^{n}x_{i}^{2}\right),\quad-1\le x_{i}\le1$} & {\scriptsize{}$n=4,16$}\tabularnewline
\hline 
{\footnotesize{}Griewank2} & {\scriptsize{}$f(x)=1+\frac{1}{200}\sum_{i=1}^{2}x_{i}^{2}-\prod_{i=1}^{2}\frac{\cos(x_{i})}{\sqrt{(i)}}$} & {\scriptsize{}2}\tabularnewline
\hline 
{\footnotesize{}Griewank10} & {\scriptsize{}f$(x)=1+\frac{1}{200}\sum_{i=1}^{10}x_{i}^{2}-\prod_{i=1}^{10}\frac{\cos(x_{i})}{\sqrt{(i)}}$} & {\scriptsize{}10}\tabularnewline
\hline 
{\footnotesize{}Gkls\citep{Gaviano}} & {\scriptsize{}$f(x)=\mbox{Gkls}(x,n,w)$} & {\scriptsize{}$n=2,3\ w=50,100$}\tabularnewline
\hline 
{\footnotesize{}Hansen} & {\scriptsize{}$f(x)=\sum_{i=1}^{5}i\cos\left[(i-1)x_{1}+i\right]\sum_{j=1}^{5}j\cos\left[(j+1)x_{2}+j\right]$} & {\scriptsize{}2}\tabularnewline
\hline 
{\footnotesize{}Hartman3} & {\scriptsize{}$f(x)=-\sum_{i=1}^{4}c_{i}\exp\left(-\sum_{j=1}^{3}a_{ij}\left(x_{j}-p_{ij}\right)^{2}\right)$} & {\scriptsize{}3}\tabularnewline
\hline 
{\footnotesize{}Hartman6} & {\scriptsize{}$f(x)=-\sum_{i=1}^{4}c_{i}\exp\left(-\sum_{j=1}^{6}a_{ij}\left(x_{j}-p_{ij}\right)^{2}\right)$} & {\scriptsize{}6}\tabularnewline
\hline 
{\footnotesize{}High Elliptic} & {\scriptsize{}$f(x)=\sum_{i=1}^{n}\left(10^{6}\right)^{\frac{i-1}{n-1}}x_{i}^{2}$} & {\scriptsize{}10}\tabularnewline
\hline 
{\footnotesize{}Potential\citep{Lennard} } & {\scriptsize{}$V_{LJ}(r)=4\epsilon\left[\left(\frac{\sigma}{r}\right)^{12}-\left(\frac{\sigma}{r}\right)^{6}\right]$} & {\scriptsize{}$n=9,15,21,30$}\tabularnewline
\hline 
{\footnotesize{}Rastrigin} & {\scriptsize{}$f(x)=x_{1}^{2}+x_{2}^{2}-\cos(18x_{1})-\cos(18x_{2})$} & {\scriptsize{}2}\tabularnewline
\hline 
{\footnotesize{}Shekel5} & {\scriptsize{}$f(x)=-\sum_{i=1}^{5}\frac{1}{(x-a_{i})(x-a_{i})^{T}+c_{i}}$} & {\scriptsize{}4}\tabularnewline
\hline 
{\footnotesize{}Shekel7} & {\scriptsize{}$f(x)=-\sum_{i=1}^{7}\frac{1}{(x-a_{i})(x-a_{i})^{T}+c_{i}}$} & {\scriptsize{}4}\tabularnewline
\hline 
{\footnotesize{}Shekel10} & {\scriptsize{}$f(x)=-\sum_{i=1}^{10}\frac{1}{(x-a_{i})(x-a_{i})^{T}+c_{i}}$} & {\scriptsize{}4}\tabularnewline
\hline 
{\footnotesize{}Sinusoidal\citep{Zabinsky}} & {\scriptsize{}$f(x)=-\left(2.5\prod_{i=1}^{n}\sin\left(x_{i}-z\right)+\prod_{i=1}^{n}\sin\left(5\left(x_{i}-z\right)\right)\right),\quad0\le x_{i}\le\pi$} & {\scriptsize{}$n=4,8$}\tabularnewline
\hline 
{\footnotesize{}Test2N} & {\scriptsize{}$f(x)=\frac{1}{2}\sum_{i=1}^{n}x_{i}^{4}-16x_{i}^{2}+5x_{i}$} & {\scriptsize{}$n=4,9$}\tabularnewline
\hline 
{\footnotesize{}Test30N} & {\scriptsize{}$\frac{1}{10}\sin^{2}\left(3\pi x_{1}\right)\sum_{i=2}^{n-1}\left(\left(x_{i}-1\right)^{2}\left(1+\sin^{2}\left(3\pi x_{i+1}\right)\right)\right)+\left(x_{n}-1\right)^{2}\left(1+\sin^{2}\left(2\pi x_{n}\right)\right)$} & {\scriptsize{}$n=3,4$}\tabularnewline
\hline 
\end{tabular}
\par\end{centering}
\caption{The test functions used in the conducted experiments.\label{tab:The-test-functions}}

\end{table}


\subsection{Experimental results}

All tables presented here demonstrate the average function calls obtained
for each test problem. The average function can be considered as a
measure of the speed of each method, since each function has a different
complexity and therefore, the average execution time would not be
a good indication of the computational time required. In the first
set of experiments we examined if the presence of the suggested stopping
rule, which is the combination of a series of stopping rules, affects
the optimization techniques without parallelization. In Table \ref{tab:DE},
we observe the performance of all termination rules for the DE method.
From these, both statistical and quantitative comparisons arise, depicted
in Figure \ref{fig:DE_stat}. 

\begin{table}[H]
\centering{}\caption{Average function calls for the Differential evolution method, using
different different termination rules\label{tab:DE}}
\begin{tabular}{|l|c|c|c|c|c|}
\hline 
\textbf{Problem} & \textbf{ITERATION} & \textbf{BEST} & \textbf{MEAN} & \textbf{DOUBLEBOX} & \textbf{ALL}\tabularnewline
\hline 
\hline 
\textbf{BF1} & 32561 & 6761 & 10511 & 10369 & 6114\tabularnewline
\hline 
\textbf{BF2} & 30339 & 7693 & 9222 & 11524 & 7560\tabularnewline
\hline 
\textbf{BRANIN} & 21650 & 4639 & 10982 & 4907 & 4289\tabularnewline
\hline 
\textbf{CAMEL} & 27444 & 6116 & 13290 & 9905 & 5940\tabularnewline
\hline 
\textbf{CIGAR10} & 31854 & 7647 & 4348 & 21116 & 3111\tabularnewline
\hline 
\textbf{CM4} & 31504 & 7913 & 4175 & 11851 & 4079\tabularnewline
\hline 
\textbf{DISCUS10} & 25477 & 5191 & 2042 & 13614 & 2034\tabularnewline
\hline 
\textbf{EASOM} & 19591 & 1917 & 15103 & 1721 & 1721\tabularnewline
\hline 
\textbf{ELP10} & 25870 & 6046 & 24675 & 12519 & 6031\tabularnewline
\hline 
\textbf{EXP4} & 27134 & 5216 & 18151 & 6385 & 5040\tabularnewline
\hline 
\textbf{EXP16} & 28161 & 5588 & 27387 & 7552 & 5339\tabularnewline
\hline 
\textbf{GKLS250} & 25362 & 5227 & 2641 & 8379 & 2641\tabularnewline
\hline 
\textbf{GKLS350} & 24645 & 5624 & 3298 & 17437 & 3206\tabularnewline
\hline 
\textbf{GRIEWANK2} & 29342 & 8027 & 10458 & 14756 & 6915\tabularnewline
\hline 
\textbf{GRIEWANK10} & 38706 & 9664 & 37839 & 17312 & 9539\tabularnewline
\hline 
\textbf{POTENTIAL3} & 26013 & 5278 & 24823 & 13871 & 5256\tabularnewline
\hline 
\textbf{PONTENTIAL5} & 32876 & 8225 & 32439 & 20828 & 7742\tabularnewline
\hline 
\textbf{PONTENTIAL6} & 35041 & 8467 & 34946 & 19169 & 8197\tabularnewline
\hline 
\textbf{PONTENTIAL10} & 41410 & 10330 & 41300 & 26308 & 10643\tabularnewline
\hline 
\textbf{HANSEN} & 23069 & 5219 & 23050 & 14245 & 4263\tabularnewline
\hline 
\textbf{HARTMAN3} & 24165 & 4613 & 17966 & 7333 & 4566\tabularnewline
\hline 
\textbf{HARTMAN6} & 25963 & 5734 & 17109 & 9354 & 5550\tabularnewline
\hline 
\textbf{RASTRIGIN} & 29501 & 5912 & 17240 & 8987 & 5999\tabularnewline
\hline 
\textbf{ROSENBROCK8} & 34964 & 8503 & 34429 & 20980 & 8051\tabularnewline
\hline 
\textbf{ROSENBROCK16} & 41395 & 10052 & 41335 & 26156 & 8307\tabularnewline
\hline 
\textbf{SHEKEL5} & 25810 & 6064 & 25800 & 19877 & 5933\tabularnewline
\hline 
\textbf{SHEKEL7} & 26177 & 5597 & 26168 & 20274 & 5677\tabularnewline
\hline 
\textbf{SHEKEL10} & 26489 & 6037 & 26439 & 22960 & 5100\tabularnewline
\hline 
\textbf{SINU4} & 24646 & 5751 & 24620 & 12511 & 4776\tabularnewline
\hline 
\textbf{SINU8} & 25355 & 6340 & 25477 & 18741 & 4980\tabularnewline
\hline 
\textbf{TEST2N4} & 25987 & 5848 & 24947 & 11640 & 5727\tabularnewline
\hline 
\textbf{TEST2N9} & 26698 & 7303 & 26571 & 16165 & 6288\tabularnewline
\hline 
\textbf{TEST30N3} & 25560 & 3169 & 8395 & 8134 & 2869\tabularnewline
\hline 
\textbf{TEST30N4} & 25021 & 2841 & 10381 & 7853 & 2714\tabularnewline
\hline 
\textbf{Total} & \textbf{965780} & \textbf{214552} & \textbf{677557} & \textbf{474733} & \textbf{186197}\tabularnewline
\hline 
\end{tabular}
\end{table}

\begin{figure}[H]
\centering{}\includegraphics{3}\caption{Statistical comparison of function calls with different termination
rule of differential evolution\label{fig:DE_stat}}
\end{figure}
As it is evident, the combination of the stopping rules reduces the
required number of function calls that are required to obtain the
global minimum by the DE method. The proposed termination scheme outperforms
by a percentage that varies from 13\% to 73\% the other termination
rules used here. Subsequently, the same experiment was conducted for
the PSO method and the results are shown in Table \ref{tab:PSO} and
the statistical comparison is outlined graphically in Figure \ref{fig:PSO_stat}.

\begin{table}[H]
\centering{}\caption{Particle swarm optimization: Function calls with different termination
rules\label{tab:PSO}}
\begin{tabular}{|l|c|c|c|c|c|}
\hline 
\textbf{Problem} & \textbf{ITERATION} & \textbf{BEST} & \textbf{MEAN} & \textbf{DOUBLEBOX} & \textbf{ALL}\tabularnewline
\hline 
\hline 
\textbf{BF1} & 33113 & 3418 & 3127 & 3122 & 3005\tabularnewline
\hline 
\textbf{BF2} & 32804 & 3265 & 2995 & 2919 & 2889\tabularnewline
\hline 
\textbf{BRANIN} & 27762 & 2601 & 2549 & 2431 & 2417\tabularnewline
\hline 
\textbf{CAMEL} & 28511 & 2801 & 2675 & 2547 & 2547\tabularnewline
\hline 
\textbf{CIGAR10} & 39038 & 3987 & 3674 & 3638 & 3638\tabularnewline
\hline 
\textbf{CM4} & 33051 & 3863 & 3299 & 3144 & 3144\tabularnewline
\hline 
\textbf{DISCUS10} & 29548 & 2611 & 2409 & 2405 & 2405\tabularnewline
\hline 
\textbf{EASOM} & 26251 & 2420 & 2232 & 2232 & 2232\tabularnewline
\hline 
\textbf{ELP10} & 1907 & 1705 & 1741 & 1586 & 1586\tabularnewline
\hline 
\textbf{EXP4} & 28364 & 2751 & 2558 & 2558 & 2558\tabularnewline
\hline 
\textbf{EXP16} & 28536 & 2898 & 2676 & 2676 & 2676\tabularnewline
\hline 
\textbf{GKLS250} & 27495 & 2796 & 2553 & 2422 & 2422\tabularnewline
\hline 
\textbf{GKLS350} & 28350 & 2947 & 2645 & 3833 & 2558\tabularnewline
\hline 
\textbf{GRIEWANK2} & 32596 & 4405 & 2839 & 4282 & 2820\tabularnewline
\hline 
\textbf{GRIEWANK10} & 37070 & 5327 & 4561 & 4797 & 4248\tabularnewline
\hline 
\textbf{POTENTIAL3} & 32890 & 3289 & 3231 & 3222 & 3170\tabularnewline
\hline 
\textbf{PONTENTIAL5} & 50595 & 4926 & 4881 & 7282 & 4730\tabularnewline
\hline 
\textbf{PONTENTIAL6} & 62076 & 6699 & 5537 & 7534 & 5077\tabularnewline
\hline 
\textbf{PONTENTIAL10} & 78807 & 10582 & 6737 & 29865 & 6539\tabularnewline
\hline 
\textbf{HANSEN} & 29527 & 3788 & 2687 & 3062 & 2587\tabularnewline
\hline 
\textbf{HARTMAN3} & 29673 & 2807 & 2743 & 2550 & 2550\tabularnewline
\hline 
\textbf{HARTMAN6} & 31739 & 3081 & 2915 & 2809 & 2809\tabularnewline
\hline 
\textbf{RASTRIGIN} & 32717 & 3558 & 2990 & 2829 & 2829\tabularnewline
\hline 
\textbf{ROSENBROCK8} & 33115 & 4279 & 4263 & 4036 & 3969\tabularnewline
\hline 
\textbf{ROSENBROCK16} & 35519 & 5761 & 5432 & 5170 & 5170\tabularnewline
\hline 
\textbf{SHEKEL5} & 29461 & 3132 & 2823 & 12476 & 2816\tabularnewline
\hline 
\textbf{SHEKEL7} & 30003 & 3064 & 2855 & 11520 & 2856\tabularnewline
\hline 
\textbf{SHEKEL10} & 30043 & 3128 & 2942 & 14786 & 2866\tabularnewline
\hline 
\textbf{SINU4} & 27786 & 3047 & 2740 & 2673 & 2657\tabularnewline
\hline 
\textbf{SINU8} & 28129 & 3177 & 2824 & 4113 & 2828\tabularnewline
\hline 
\textbf{TEST2N4} & 28950 & 3037 & 2839 & 2681 & 2681\tabularnewline
\hline 
\textbf{TEST2N9} & 30368 & 4188 & 3106 & 3105 & 3067\tabularnewline
\hline 
\textbf{TEST30N3} & 27643 & 2943 & 3130 & 2762 & 2762\tabularnewline
\hline 
\textbf{TEST30N4} & 27836 & 3202 & 3185 & 2915 & 2915\tabularnewline
\hline 
\textbf{Total} & \textbf{1111273} & \textbf{125483} & \textbf{110393} & \textbf{169982} & \textbf{106023}\tabularnewline
\hline 
\end{tabular}
\end{table}

\begin{figure}[H]
\begin{centering}
\includegraphics{5}
\par\end{centering}
\caption{Statistical comparison of function calls with different termination
rule of particle swarm optimization\label{fig:PSO_stat}}
\end{figure}
Once more, the proposed termination scheme outperforms the other stopping
rules in almost every test function. Likewise, the same experiment
was carried out for the Multistart method and the results are depicted
in Table \ref{tab:Multistart} and the statistical comparison is shown
in Figure \ref{fig:Multistart_stat}.

\begin{table}[H]
\begin{centering}
\caption{Multistart: Function calls with different termination rules\label{tab:Multistart}}
\begin{tabular}{|l|c|c|c|c|c|}
\hline 
\textbf{Problem} & \textbf{ITERATION} & \textbf{BEST} & \textbf{MEAN} & \textbf{DOUBLEBOX} & \textbf{ALL}\tabularnewline
\hline 
\hline 
\textbf{BF1} & 479374 & 53873 & 479374 & 50762 & 50762\tabularnewline
\hline 
\textbf{BF2} & 276214 & 37670 & 239220 & 35711 & 35711\tabularnewline
\hline 
\textbf{BRANIN} & 107283 & 11063 & 11046 & 10521 & 10521\tabularnewline
\hline 
\textbf{CAMEL} & 163508 & 16083 & 153559 & 15255 & 15255\tabularnewline
\hline 
\textbf{CIGAR10} & 59097 & 14937 & 14937 & 14697 & 14697\tabularnewline
\hline 
\textbf{CM4} & 726033 & 62236 & 62192 & 58581 & 58581\tabularnewline
\hline 
\textbf{DISCUS10} & 50456 & 6296 & 6296 & 6056 & 6056\tabularnewline
\hline 
\textbf{EASOM} & 55130 & 5412 & 55130 & 51132 & 5412\tabularnewline
\hline 
\textbf{ELP10} & 58917 & 14757 & 14981 & 14517 & 14517\tabularnewline
\hline 
\textbf{EXP4} & 54174 & 10022 & 54174 & 9774 & 9774\tabularnewline
\hline 
\textbf{EXP16} & 54680 & 10528 & 54680 & 10280 & 10280\tabularnewline
\hline 
\textbf{GKLS250} & 58547 & 6202 & 47948 & 5908 & 5908\tabularnewline
\hline 
\textbf{GKLS350} & 81988 & 7560 & 42145 & 7087 & 7087\tabularnewline
\hline 
\textbf{GRIEWANK2} & 188429 & 19179 & 188429 & 17877 & 17877\tabularnewline
\hline 
\textbf{GRIEWANK10} & 620223 & 65206 & 567153 & 61818 & 61818\tabularnewline
\hline 
\textbf{POTENTIAL3} & 126632 & 17773 & 126632 & 17161 & 17161\tabularnewline
\hline 
\textbf{PONTENTIAL5} & 226873 & 31347 & 226873 & 30249 & 30249\tabularnewline
\hline 
\textbf{PONTENTIAL6} & 247144 & 35457 & 247144 & 33883 & 33883\tabularnewline
\hline 
\textbf{PONTENTIAL10} & 283230 & 44957 & 283230 & 43618 & 43618\tabularnewline
\hline 
\textbf{HANSEN} & 201543 & 18568 & 201543 & 17541 & 17541\tabularnewline
\hline 
\textbf{HARTMAN3} & 162934 & 17395 & 162934 & 16562 & 16562\tabularnewline
\hline 
\textbf{HARTMAN6} & 179073 & 22010 & 179073 & 21015 & 21015\tabularnewline
\hline 
\textbf{RASTRIGIN} & 275610 & 24696 & 275610 & 23015 & 23015\tabularnewline
\hline 
\textbf{ROSENBROCK8} & 62010 & 17850 & 17850 & 17610 & 17610\tabularnewline
\hline 
\textbf{ROSENBROCK16} & 69460 & 25300 & 25300 & 25060 & 25060\tabularnewline
\hline 
\textbf{SHEKEL5} & 151648 & 17725 & 151648 & 16972 & 16972\tabularnewline
\hline 
\textbf{SHEKEL7} & 154373 & 17894 & 154373 & 17127 & 17127\tabularnewline
\hline 
\textbf{SHEKEL10} & 159825 & 17931 & 159825 & 17135 & 17135\tabularnewline
\hline 
\textbf{SINU4} & 155634 & 16429 & 155634 & 15647 & 15647\tabularnewline
\hline 
\textbf{SINU8} & 170167 & 19519 & 170167 & 18674 & 18674\tabularnewline
\hline 
\textbf{TEST2N4} & 154092 & 16486 & 154092 & 15806 & 15806\tabularnewline
\hline 
\textbf{TEST2N9} & 160651 & 22035 & 160651 & 20716 & 20716\tabularnewline
\hline 
\textbf{TEST30N3} & 161628 & 16934 & 161628 & 16145 & 16145\tabularnewline
\hline 
\textbf{TEST30N4} & 159753 & 16686 & 159753 & 15907 & 15907\tabularnewline
\hline 
\textbf{Total} & \textbf{6296333} & \textbf{758016} & \textbf{5165224} & \textbf{769819} & \textbf{724099}\tabularnewline
\hline 
\end{tabular}
\par\end{centering}
\end{table}

\begin{figure}[H]
\begin{centering}
\includegraphics{7}\caption{Statistical comparison of function calls with different termination
rule of multistart\label{fig:Multistart_stat}}
\par\end{centering}
\end{figure}
 Observing the tables with the corresponding statistical and quantitative
comparisons, it is evident that the calls to the objective function
for the proposed termination rule are fewer
than any other rule in any method. 

The proposed termination rule was also applied to the current parallel
optimization technique for the test function described previously.
The experimental results for the so - called \emph{DoubleBox} stopping
rule and the given parallel method are show in Table \ref{tab:parallelDoublebox}.
The columns in the table stand for the following:
\begin{enumerate}
\item Column Problem denotes the test function used.
\item Column 1x500 denotes the application of the proposed technique with
one processing unit and 500 particles.
\item Column 2x250 stands for the usage of two processing units. In each
unit, the number of particles was set to 250.
\item Column 5x100 denotes the incorporation of 5 processing units into
the proposed method. In each processing unit the number of particles
was set to 100.
\item Column 10x50 stands for the usage of 10 processing units. In each
processing unit, the number of particles was set to 50.
\end{enumerate}
\begin{table}[H]
\centering{}\caption{Experiments using the proposed optimization technique and the Doublebox
stopping rule. The experiment was performed on the test functions
described previously. Numbers in cells denote average function calls.\label{tab:parallelDoublebox}}
\begin{tabular}{|c|c|c|c|c|}
\hline 
\textbf{Problem} & \textbf{1x500} & \textbf{2x250} & \textbf{5x100} & \textbf{10x50}\tabularnewline
\hline 
\hline 
\textbf{BF1} & 36018 & 28282 & 18419 & 10859\tabularnewline
\hline 
\textbf{BF2} & 42191 & 37555 & 16568 & 10710\tabularnewline
\hline 
\textbf{BRANIN} & 45165 & 45726 & 45757 & 34381\tabularnewline
\hline 
\textbf{CAMEL} & 54786 & 54782 & 54253 & 53839\tabularnewline
\hline 
\textbf{CIGAR10} & 18387 & 18300 & 18307 & 11950\tabularnewline
\hline 
\textbf{CM4} & 60796 & 60501 & 60287 & 60200\tabularnewline
\hline 
\textbf{DISCUS10} & 25115 & 24586 & 14565 & 8957\tabularnewline
\hline 
\textbf{EASOM} & 40376 & 40211 & 40100 & 40099\tabularnewline
\hline 
\textbf{ELP10} & 20193 & 14745 & 11211 & 7691\tabularnewline
\hline 
\textbf{EXP4} & 54779 & 54676 & 54416 & 54412\tabularnewline
\hline 
\textbf{EXP16} & 52768 & 52205 & 52563 & 52560\tabularnewline
\hline 
\textbf{GKLS250} & 51980 & 51240 & 51233 & 51231\tabularnewline
\hline 
\textbf{GKLS350} & 48577 & 48761 & 48702 & 4100\tabularnewline
\hline 
\textbf{GRIEWANK2} & 56767 & 56690 & 51170 & 26076\tabularnewline
\hline 
\textbf{GRIEWANK10} & 20941 & 20721 & 14870 & 11765\tabularnewline
\hline 
\textbf{POTENTIAL3} & 49418 & 49800 & 49109 & 49002\tabularnewline
\hline 
\textbf{PONTENTIAL5} & 59827 & 59656 & 59367 & 59322\tabularnewline
\hline 
\textbf{PONTENTIAL6} & 62720 & 62398 & 62294 & 62280\tabularnewline
\hline 
\textbf{PONTENTIAL10} & 72899 & 72745 & 72943 & 72578\tabularnewline
\hline 
\textbf{HANSEN} & 45064 & 45689 & 45510 & 45424\tabularnewline
\hline 
\textbf{HARTMAN3} & 47307 & 47112 & 47111 & 47100\tabularnewline
\hline 
\textbf{HARTMAN6} & 49370 & 49302 & 49312 & 49222\tabularnewline
\hline 
\textbf{RASTRIGIN} & 57617 & 56577 & 56513 & 56400\tabularnewline
\hline 
\textbf{ROSENBROCK8} & 22997 & 22903 & 14139 & 14001\tabularnewline
\hline 
\textbf{POSENBROCK16} & 42117 & 34856 & 21392 & 13722\tabularnewline
\hline 
\textbf{SHEKEL5} & 50631 & 50198 & 50233 & 44662\tabularnewline
\hline 
\textbf{SHEKEL7} & 51212 & 51200 & 51200 & 45610\tabularnewline
\hline 
\textbf{SHEKEL10} & 51741 & 51607 & 51587 & 47761\tabularnewline
\hline 
\textbf{SINU4} & 48140 & 48100 & 48111 & 48099\tabularnewline
\hline 
\textbf{SINU8} & 48761 & 48434 & 48423 & 48204\tabularnewline
\hline 
\textbf{TEST2N4} & 48904 & 48307 & 48354 & 48354\tabularnewline
\hline 
\textbf{TEST2N9} & 48838 & 48555 & 48600 & 48335\tabularnewline
\hline 
\textbf{TEST30N3} & 51640 & 51307 & 51194 & 51143\tabularnewline
\hline 
\textbf{TEST30N4} & 48992 & 48549 & 48540 & 48487\tabularnewline
\hline 
\textbf{Total} & \textbf{1587034} & \textbf{1556276} & \textbf{1476353} & \textbf{1338536}\tabularnewline
\hline 
\end{tabular}
\end{table}
In this experiment, the total number of particles was set to 500,
in order to have credibility in the values recorded. The results indicate
that the increase in processing units may reduce the required number
of function calls. The same series of experiments was also conducted
using the so - called \emph{mean fitness} termination rule, that has
been described in equation \ref{eq:mean}. The obtained results are
presented in Table \ref{tab:parallelBest}. 

\begin{table}[H]
\begin{centering}
\caption{Experiments using the proposed optimization technique and the mean
- fitness stopping rule. Numbers in cells denote average function
calls.\label{tab:parallelBest}}
\par\end{centering}
\centering{}%
\begin{tabular}{|c|c|c|c|c|}
\hline 
\textbf{Problems} & \textbf{1x500} & \textbf{2x250} & \textbf{5x100} & \textbf{10x50}\tabularnewline
\hline 
\hline 
\textbf{BF1} & 49714 & 41284 & 41089 & 19596\tabularnewline
\hline 
\textbf{BF2} & 58272 & 46270 & 41324 & 15878\tabularnewline
\hline 
\textbf{BRANIN} & 45165 & 45665 & 44894 & 7268\tabularnewline
\hline 
\textbf{CAMEL} & 54786 & 54989 & 54651 & 10295\tabularnewline
\hline 
\textbf{CIGAR10} & 63026 & 62206 & 60277 & 27217\tabularnewline
\hline 
\textbf{CM4} & 60796 & 60333 & 40072 & 12977\tabularnewline
\hline 
\textbf{DISCUS10} & 50988 & 50690 & 36151 & 14040\tabularnewline
\hline 
\textbf{EASOM} & 40376 & 34178 & 21355 & 7377\tabularnewline
\hline 
\textbf{ELP10} & 50397 & 16787 & 9334 & 7809\tabularnewline
\hline 
\textbf{EXP4} & 54749 & 54445 & 29812 & 11024\tabularnewline
\hline 
\textbf{EXP16} & 52768 & 51233 & 24782 & 10722\tabularnewline
\hline 
\textbf{GKLS250} & 51980 & 50333 & 37696 & 9641\tabularnewline
\hline 
\textbf{GKLS350} & 48577 & 48442 & 28420 & 9337\tabularnewline
\hline 
\textbf{GRIEWANK2} & 56767 & 40471 & 30632 & 12356\tabularnewline
\hline 
\textbf{GRIEWANK10} & 68439 & 58410 & 53818 & 30317\tabularnewline
\hline 
\textbf{POTENTIAL3} & 49418 & 49328 & 33657 & 10140\tabularnewline
\hline 
\textbf{PONTENTIAL5} & 59827 & 58259 & 34436 & 11515\tabularnewline
\hline 
\textbf{PONTENTIAL6} & 62720 & 62715 & 34158 & 14417\tabularnewline
\hline 
\textbf{PONTENTIAL10} & 72899 & 72800 & 64964 & 16780\tabularnewline
\hline 
\textbf{HANSEN} & 45064 & 45582 & 32013 & 9318\tabularnewline
\hline 
\textbf{HARTMAN3} & 47307 & 47983 & 31033 & 8899\tabularnewline
\hline 
\textbf{HARTMAN6} & 49370 & 47983 & 27884 & 9602\tabularnewline
\hline 
\textbf{RASTRIGIN} & 54637 & 53551 & 33095 & 10169\tabularnewline
\hline 
\textbf{ROSENBROCK8} & 67885 & 57314 & 46961 & 25383\tabularnewline
\hline 
\textbf{POSENBROCK16} & 77207 & 66172 & 59098 & 41461\tabularnewline
\hline 
\textbf{SHEKEL5} & 50631 & 49800 & 32025 & 10389\tabularnewline
\hline 
\textbf{SHEKEL7} & 51212 & 50630 & 29136 & 11193\tabularnewline
\hline 
\textbf{SHEKEL10} & 51741 & 50321 & 33462 & 11200\tabularnewline
\hline 
\textbf{SINU4} & 48140 & 52788 & 32412 & 9721\tabularnewline
\hline 
\textbf{SINU8} & 48761 & 55864 & 32685 & 10410\tabularnewline
\hline 
\textbf{TEST2N4} & 48904 & 48800 & 28959 & 8643\tabularnewline
\hline 
\textbf{TEST2N9} & 48838 & 47001 & 36281 & 8085\tabularnewline
\hline 
\textbf{TEST30N3} & 51514 & 50100 & 39828 & 11699\tabularnewline
\hline 
\textbf{TEST30N4} & 48992 & 46875 & 42564 & 11954\tabularnewline
\hline 
\textbf{Total} & \textbf{1841867} & \textbf{1729602} & \textbf{1258958} & \textbf{456832}\tabularnewline
\hline 
\end{tabular}
\end{table}
And in this series of experiments, the columns of the table retain
the same meaning as in Table \ref{tab:parallelDoublebox}. Furthermore,
once again, it is observed that the increase in the number of computing
units significantly reduces the required number of function calls
to find the global minimum. Additionally, there is a significant reduction
in the number of function calls required to be compared to the previous
termination rule. Finally, the proposed termination rule is utilized
in the parallel optimization technique and the experimental results
are outlined in Table \ref{tab:parallelProposed}.

\begin{table}[H]
\begin{centering}
\caption{Experiments using the proposed optimization technique and proposed
stopping rule. Numbers in cells denote average function calls.\label{tab:parallelProposed}}
\par\end{centering}
\centering{}%
\begin{tabular}{|c|c|c|c|c|}
\hline 
\textbf{Problems} & \textbf{1x500} & \textbf{2x250} & \textbf{5x100} & \textbf{10x50}\tabularnewline
\hline 
\hline 
\textbf{BF1} & 10839 & 6543 & 6375 & 5933\tabularnewline
\hline 
\textbf{BF2} & 9838 & 6317 & 5978 & 5716\tabularnewline
\hline 
\textbf{BRANIN} & 5787 & 5003 & 5076 & 4936\tabularnewline
\hline 
\textbf{CAMEL} & 12204 & 5889 & 5703 & 5434\tabularnewline
\hline 
\textbf{CIGAR10} & 7631 & 6575 & 6582 & 6500\tabularnewline
\hline 
\textbf{CM4} & 16222 & 8935 & 6837 & 6392\tabularnewline
\hline 
\textbf{DISCUS10} & 5997 & 5039 & 5001 & 5252\tabularnewline
\hline 
\textbf{EASOM} & 4701 & 4791 & 4772 & 4751\tabularnewline
\hline 
\textbf{ELP10} & 6230 & 5363 & 5471 & 5205\tabularnewline
\hline 
\textbf{EXP4} & 10277 & 5542 & 5620 & 5533\tabularnewline
\hline 
\textbf{EXP16} & 9577 & 5518 & 5544 & 5532\tabularnewline
\hline 
\textbf{GKLS250} & 13328 & 5993 & 5205 & 5011\tabularnewline
\hline 
\textbf{GKLS350} & 10131 & 7373 & 5094 & 4869\tabularnewline
\hline 
\textbf{GRIEWANK2} & 8504 & 8284 & 6371 & 5761\tabularnewline
\hline 
\textbf{GRIEWANK10} & 8535 & 7585 & 7229 & 6922\tabularnewline
\hline 
\textbf{POTENTIAL3} & 11338 & 5666 & 5830 & 5695\tabularnewline
\hline 
\textbf{PONTENTIAL5} & 12414 & 7310 & 7126 & 6996\tabularnewline
\hline 
\textbf{PONTENTIAL6} & 14760 & 8450 & 7477 & 7561\tabularnewline
\hline 
\textbf{PONTENTIAL10} & 17729 & 11928 & 10431 & 9338\tabularnewline
\hline 
\textbf{HANSEN} & 8379 & 6862 & 4883 & 4970\tabularnewline
\hline 
\textbf{HARTMAN3} & 10674 & 5395 & 5219 & 5078\tabularnewline
\hline 
\textbf{HARTMAN6} & 10841 & 5486 & 5649 & 5348\tabularnewline
\hline 
\textbf{RASTRIGIN} & 13657 & 8283 & 5850 & 5292\tabularnewline
\hline 
\textbf{ROSENBROCK8} & 9456 & 6751 & 7232 & 6979\tabularnewline
\hline 
\textbf{POSENBROCK16} & 9853 & 7993 & 8377 & 8086\tabularnewline
\hline 
\textbf{SHEKEL5} & 10959 & 5749 & 5810 & 5884\tabularnewline
\hline 
\textbf{SHEKEL7} & 10651 & 5828 & 5843 & 5698\tabularnewline
\hline 
\textbf{SHEKEL10} & 1533 & 5863 & 6112 & 5588\tabularnewline
\hline 
\textbf{SINU4} & 10291 & 6427 & 5371 & 5251\tabularnewline
\hline 
\textbf{SINU8} & 9993 & 7139 & 6274 & 5583\tabularnewline
\hline 
\textbf{TEST2N4} & 10242 & 7516 & 5573 & 5129\tabularnewline
\hline 
\textbf{TEST2N9} & 12998 & 10137 & 6112 & 5566\tabularnewline
\hline 
\textbf{TEST30N3} & 6228 & 5512 & 5402 & 5485\tabularnewline
\hline 
\textbf{TEST30N4} & 5540 & 5749 & 5750 & 5350\tabularnewline
\hline 
\textbf{Total} & \textbf{337337} & \textbf{228794} & \textbf{207179} & \textbf{198624}\tabularnewline
\hline 
\end{tabular}
\end{table}
And in this case, it is observed that the increase of parallel processing
units significantly reduces the required number of function calls,
as in the two previous termination rules. However, the proposed termination
rule dramatically reduces the number of required function calls and
consequently the computation time compared to previous termination
rules. This reduction in fact intensifies as the number of computing
nodes increases. Also, for the same set of experiments, a statistical
test (Wilcoxon test) was conducted and the it is graphically presented
in Figure \ref{fig:Wilcoxon-rank-sum-test}.

\begin{figure}[H]
\begin{centering}
\includegraphics[scale=0.75]{wilcoxon}
\par\end{centering}
\caption{Wilcoxon rank-sum test results for the comparison of different termination
rules as used in the proposed parallel optimization technique and
for 5 processing units.\label{fig:Wilcoxon-rank-sum-test}}

\end{figure}
 A p-value of less than 0.05 (two-tailed) was used to determine statistical
significance and is indicated in bold.

An additional experiment was performed in order to demonstrate the
efficiency of the proposed technique on high-dimensional functions.
For this reason, the high elliptic function was selected. The function
is defined as 
\[
f(x)=\sum_{i=1}^{n}\left(10^{6}\right)^{\frac{i-1}{n-1}}x_{i}^{2}
\]
where the $n$ stands for the dimension of the function. The results
from the application of the current parallel method to this function
when the variable $n$ varies from $n=10$ to $n=60$ are graphically
illustrated in Figure \ref{fig:elpTime}.

\begin{figure}[H]
\begin{centering}
\includegraphics[scale=0.65]{elp_parallel}
\par\end{centering}
\caption{Execution time for the proposed method when applied to the ELP problem
for different dimensions.\label{fig:elpTime}}

\end{figure}
As in the previous experiments, increasing the number of computing
units leads to a reduction in the required number of function calls
and as can be seen in this graph, this in turn leads to a reduction
in computing time.

\section{Conclusions \label{sec:Conclusions}}

In this paper, the use of mixed termination rules for global optimization
methods was thoroughly presented, and a new global optimization method
that takes full advantage of parallel computing structures was afterward
proposed. The use of mixed termination rules leads a series of computational
global optimization techniques to find the global minimum of the objective
function faster, as was also shown by the experimental results. The
termination rules exploited are based on asymptotic criteria and are
general enough to be applicable to any global optimization problem
and stochastic global optimization technique.

Furthermore, the new stopping rule was utilized in a novel stochastic
global optimization technique that involves some basic global optimization
techniques. This method is designed to be executed in parallel computation
environments. At each step of the method, a different global optimization
method is also executed on each parallel computing unit, and these
different techniques exchange optimal solutions with each other. In
current work, the global optimization techniques of Differential Evolution,
Multistart and Particle Swarm optimization were used in the processing
units, but the method can be generalized to use other stochastic optimization
techniques, such as genetic algorithms or simulated annealing. Also,
the overall method terminates with the combined termination rule proposed
here, and the experimental results performed on a series of well -
known test problems from the relevant literature seem to be very promising.

Future extensions of the current work may include the incorporation
of more advanced termination rules as well as the execution of alternative
global optimization methods in the processing units, such as genetic
algorithms or variants of the simulated annealing.

\vspace{6pt}

$ $

\authorcontributions{V.C., I.G.T and A.M.G conceived the idea and methodology and supervised
the technical part regarding the software. V.C. conducted the experiments.
A.M.G. performed the statistical analysis. I.G.T. and all other authors
prepared the manuscript. All authors have read and agreed to the published
version of the manuscript.}

\institutionalreview{Not applicable.}

\informedconsent{Not applicable.}

\dataavailability{Not applicable.}

\acknowledgments{This research has been financed by the European Union : Next Generation
EU through the Program Greece 2.0 National Recovery and Resilience
Plan , under the call RESEARCH -- CREATE -- INNOVATE, project name
“iCREW: Intelligent small craft simulator for advanced crew training
using Virtual Reality techniques\textquotedbl{} (project code:TAEDK-06195).}

\conflictsofinterest{The authors have no conflicts of interest to declare.}

\appendixtitles{no}

\appendixstart{}

\appendix

\begin{adjustwidth}{-\extralength}{0cm}{}


\reftitle{References}
\begin{thebibliography}{99}
\bibitem{go_physics1}M. Honda, Application of genetic algorithms
to modelings of fusion plasma physics, Computer Physics Communications
\textbf{231}, pp. 94-106, 2018.

\bibitem{go_physics2}X.L. Luo, J. Feng, H.H. Zhang, A genetic algorithm
for astroparticle physics studies, Computer Physics Communications
\textbf{250}, 106818, 2020.

\bibitem{go_physics3}T.M. Aljohani, A.F. Ebrahim, O. Mohammed, Single
and Multiobjective Optimal Reactive Power Dispatch Based on Hybrid
Artificial Physics--Particle Swarm Optimization, Energies \textbf{12},
2333, 2019.

\bibitem{go_chemistry1}P.M. Pardalos, D. Shalloway, G. Xue, Optimization
methods for computing global minima of nonconvex potential energy
functions, Journal of Global Optimization \textbf{4}, pp. 117-133,
1994.

\bibitem{go_chemistry2}A. Liwo, J. Lee, D.R. Ripoll, J. Pillardy,
H. A. Scheraga, Protein structure prediction by global optimization
of a potential energy function, Biophysics \textbf{96}, pp. 5482-5485,
1999.

\bibitem{go_chemistry3}J. An, G.He, F. Qin, R. Li, Z. Huang, A new
framework of global sensitivity analysis for the chemical kinetic
model using PSO-BPNN, Computers \& Chemical Engineering \textbf{112},
pp. 154-164, 2018.

\bibitem{go_econ1}Zwe-Lee Gaing, Particle swarm optimization to solving
the economic dispatch considering the generator constraints, IEEE
Transactions on \textbf{18} Power Systems, pp. 1187-1195, 2003.

\bibitem{go_econ2}M. Basu, A simulated annealing-based goal-attainment
method for economic emission load dispatch of fixed head hydrothermal
power systems, International Journal of Electrical Power \& Energy
Systems \textbf{27}, pp. 147-153, 2005.

\bibitem{go_med1}Y. Cherruault, Global optimization in biology and
medicine, Mathematical and Computer Modelling \textbf{20}, pp. 119-132,
1994.

\bibitem{go_med2}Eva K. Lee, Large-Scale Optimization-Based Classification
Models in Medicine and Biology, Annals of Biomedical Engineering \textbf{35},
pp 1095-1109, 2007.

\bibitem{interval1}M.A. Wolfe, Interval methods for global optimization,
Applied Mathematics and Computation \textbf{75}, pp. 179-206, 1996.

\bibitem{interval2}T. Csendes and D. Ratz, Subdivision Direction
Selection in Interval Methods for Global Optimization, SIAM J. Numer.
Anal. \textbf{34}, pp. 922--938, 1997. 

\bibitem{Kawachi} Kawachi, M., Ando, N. ( 1989). Genetic Algorithms
in Search, Optimization \& Machine Learning. Artificial Intelligence,
Vol.: 7, Issue: 1, pp.: 168. Doi: https://doi.org/10.11517/jjsai.7.1\_168 

\bibitem{Kirkpatrick} Kirkpatrick, S., Gelatt, C.D., Vecchi, M.P.
(1983). Optimization by simulated annealing. Science, Vol. 220, Issue:
4598, pp.: 671--680. DOI: 10.1126/science.220.4598.671 {[}CrossRef{]}
{[}PubMed{]}

\bibitem{Hashim} Hashim, F.A., Houssein, E.H., Mabrouk, M.S., Al-Atabany,W.,
Mirjalili, S. (2019). Henry gas solubility optimization: A novel physics
based algorithm. Future Generation Computer Systems: Vol.: 101, pp.:
646--667. Doi: https: 10.1016/j.future.2019.07.015

\bibitem{Rashedi} Rashedi, E., Nezamabadi-Pour, H., Saryazdi, S.
(2009). GSA: A gravitational search algorithm. Information Sciences:
Vol.: 179, pp.:2232--2248. https://doi.org/10.1016/j.ins.2009.03.004

\bibitem{Du-Wu} Du, H., Wu, X., Zhuang, J. (2006). Small-World Optimization
Algorithm for Function Optimization. In Proceedings of the International
Conference on Natural Computation, Xi’an, China, 24--28 September
2006, pp. 264--273. Doi: 10.1007/11881223\_33 

\bibitem{Goldberg} Goldberg, D.E., Holland, J.H. (1988). Genetic
Algorithms and Machine Learning. Machine Learning (3): pp.: 95--99.
Doi: 10.1023/A:1022602019183

\bibitem{Das} Das, S. , Suganthan, P.N. (2011). Differential evolution:
A survey of the state-of-the-art. IEEE Transactions on Evolutionary
Computation, Vol.: 15, Issue:1, pp.: 4--31. Doi: 10.1109/TEVC.2010.2059031

\bibitem{Charilogis} V. Charilogis , I.G. Tsoulos, A. Tzallas, E.
Karvounis, Modifications for the Differential Evolution Algorithm,
Symmetry \textbf{14}, 447, 2022.

\bibitem{Kennedy} Kennedy, J., Eberhart, R. (1995). Particle Swarm
Optimization. In Proceedings of the ICNN’95---International Conference
on Neural Networks, Perth,WA, Australia, 27 November--1 December
1995, Vol.: 4, pp. 1942--1948. Doi: 10.1109/ICNN.1995.488968

\bibitem{Charilogis2} V. Charilogis, I.G. Tsoulos, Toward an Ideal
Particle Swarm Optimizer for Multidimensional Functions, Information
\textbf{13}, 217, 2022.

\bibitem{Dorigo} Dorigo, M., Maniezzo, V., Colorni, A. (1996). Ant
system: Optimization by a colony of cooperating agents. IEEE Transactions
on Systems, Man, and Cybernetics, Part B (Cybernetics), Vol.: 26,
pp.: 29--41. Doi: 10.1109/3477.484436 

\bibitem{Yang} Yang, X.S., Gandomi, A.H. (2012). Bat algorithm: A
novel approach for global engineering optimization. Engineering Computations,
Vol.: 29, pp.: 464--483. Doi: https://doi.org/10.1108/02644401211235834

\bibitem{Mirjalili} Mirjalili, S., Lewis, A. (2016). The whale optimization
algorithm. Advances in Engineering Softwar, Vol.: 95, pp.: 51--67.
Doi: https://doi.org/10.1016/j.advengsoft.2016.01.008

\bibitem{Saremi} Saremi, S., Mirjalili, S., Lewis, A. (2017). Grasshopper
optimisation algorithm: Theory and application. Advances in Engineering
Software, Vol.: 105, pp.: 30--47.Doi: https://doi.org/10.1016/j.advengsoft.2017.01.004.

\bibitem{Rao} Rao, R.V., Savsani, V.J., Vakharia, D. (2011). Teaching--learning-based
optimization: A novel method for constrained mechanical design optimization
problems. Computer-Aided Design, Vol.: 43, pp.: 303--315. Doi: https://doi.org/10.1016/j.cad.2010.12.015

\bibitem{parallel-pso}J. F. Schutte, J. A. Reinbolt, B. J. Fregly, R.
T. Haftka, A. D. George, Parallel global optimization with the particle
swarm algorithm, International journal for Numerical methods in Engineering
\textbf{61}, pp. 2296-2315, 2004.

\bibitem{parallel-multistart}J. Larson and S.M. Wild, Asynchronously
parallel optimization solver for finding multiple minima, Mathematical
Programming Computation \textbf{10}, pp. 303-332, 2018.

\bibitem{parallel-doublepop}I.G. Tsoulos, A. Tzallas, D. Tsalikakis,
PDoublePop: An implementation of parallel genetic algorithm for function
optimization, Computer Physics Communications \textbf{209}, pp. 183-189,
2016.

\bibitem{msgpu1}R. Kamil, S. Reiji, An Efficient GPU Implementation
of a Multi-Start TSP Solver for Large Problem Instances, Proceedings
of the 14th Annual Conference Companion on Genetic and Evolutionary
Computation, pp. 1441-1442, 2012.

\bibitem{msgpu2}Van Luong T., Melab N., Talbi EG. (2011) GPU-Based
Multi-start Local Search Algorithms. In: Coello C.A.C. (eds) Learning
and Intelligent Optimization. LION 2011. Lecture Notes in Computer
Science, vol 6683. Springer, Berlin, Heidelberg. https://doi.org/10.1007/978-3-642-25566-3\_24

\bibitem{msgpu3}K. Barkalov, V. Gergel, Parallel global optimization
on GPU, J Glob Optim \textbf{66}, pp. 3--20, 2016. 

\bibitem{Low Gon} Low, Y., Gonzalez, J., Kyrola, A., Bickson,D.,
Guestrin, C., Hellerstein, J. (2010). GraphLab: A New Framework For
Parallel Machine Learning. Source: arXiv, Computer Science. 

\bibitem{Yangyang} Yangyang, L., Liu, G., Lu, G., Jiao, L., Marturi,
N. \& Shang, R. (2019). Hyper-Parameter Optimization Using MARS Surrogate
for Machine-Learning Algorithms. IEEE Transactions on Emerging Topics
in Computational Intelligence, pp(99):1-11. DOI: 10.1109/TETCI.2019.2918509. 

\bibitem{Yamashiro}Yamashiro, H. \& Nonaka, H. (2021). Estimation
of processing time using machine learning and real factory data for
optimization of parallel machine scheduling problem. ScienceDirect:
Operations Research Perspectives, Vol.:8, 2021, Doi: https://doi.org/10.1016/j.orp.2021.100196
.

\bibitem{Kim Tsai} Kim, H.S. \& Tsai, L. (2003). Design Optimization
of a Cartesian Parallel Manipulator. Journal of Mechanical Design,125(1):43-51.
Doi: https://doi.org/10.1115/1.1543977 .

\bibitem{Oh Jong} Oh, S., Jong Jang,H. \& Pedrycz, W. (2009). The
design of a fuzzy cascade controller for ball and beam system: A study
in optimization with the use of parallel genetic algorithms. ScienceDirect:
Engineering Applications of Artificial Intelligence, 22(2):261-271.
Doi: https://doi.org/10.1016/j.engappai.2008.07.003 .

\bibitem{Fatehi} Fatehi, M., Toloei, A., Zio, E., Niaki, S.T.A. \&
Kesh,B. (2023). Robust optimization of the design of monopropellant
propulsion control systems using an advanced teaching-learning-based
optimization method. ScienceDirect: Engineering Applications of Artificial
Intelligence, Vol.:126, Part: A. 

\bibitem{Cai Yang} Cai, J., Yang, H., Lai, T. \& Xu, K.(2023). Parallel
pump and chiller system optimization method for minimizing energy
consumption based on a novel multi-objective gorilla troops optimizer.
ScienceDirect: Journal Of Building Engineering, Vol.: 76, Doi: https://doi.org/10.1016/j.jobe.2023.107366

\bibitem{Yu Shahabi} Yu, Y. \& Shahabi, L. (2023). Engineering Application
of Artificial Intelligence: Optimal infrastructure in microgrids with
diverse uncertainties based on demand response, renewable energy sources
and two-stage parallel optimization algorithm. ScienceDirect. Vol.:123,
Part b. Doi: https://doi.org/10.1016/j.engappai.2023.106233 

\bibitem{Ramirez} Ramirez-Gil, F.J., Pere-Madrid, C.M., Nelli Silva,
E.C. \& Montealerge-Rubio, W. (2021). Sustainable Computing: Informatics
and Systems: Parallel computing for the topology optimization method:
Performance metrics and energy consumption analysis in multiphysics
problems. Vol.:30, Doi: https://doi.org/10.1016/j.suscom.2020.100481 

\bibitem{Tavakolan} Tavakolan, M., Mostafazadeh, F., Eirdmousa, S.J.,
Safari, A. \& Mirzai, K. (2022). A parallel computing simulation-based
multi-objective optimization framework for economic analysis of building
energy retrofit: A case study in Iran. ScienceDirect: Journal of Building
Engineering, Vol.: 45, Doi: https://doi.org/10.1016/j.jobe.2021.103485

\bibitem{Lin} Lin, G. (2020). Parallel optimization n based operational
planning to enhance the resilience of large-scale power systems. Mississippi
State University, Scholars Junction. Online: https://scholarsjunction.msstate.edu/cgi/viewcontent.cgi?article=4435\&context=td

\bibitem{Pang} Pang, M. \& Shoemaker, C.A. (2023). Comparison of
parallel optimization algorithms on computationally expensive groundwater
remediation designs. Sience of the Total Environment: 857(3), Doi:
https://doi.org/10.1016/j.scitotenv.2022.159544

\bibitem{Ezugwu} Ezugwu, A. (2023). A general Framework for Utilizing
Metaheuristic Optimization for Sustainable Unrelated Parallel Machine
Scheduling: A concise overview. Arxiv, Computer Science, Neural and
Evolutionary Computing. Doi: https://doi.org/10.48550/arXiv.2311.12802 

\bibitem{Censor} Censor, Y. \& Zenios, S. (1997). Parallel Optimization:
Theory, Algorithms and Applications. Publisher: Oxford University
Press, USAISBN: ISBN-13: 978-0195100624. DOI: 10.1093/oso/9780195100624.001.0001 

\bibitem{go_par1}E. Onbaşoğlu, L. Özdamar, Parallel simulated annealing
algorithms in global optimization. Journal of global optimization
\textbf{19}, pp. 27-50, 2001.

\bibitem{go_par2}J.F. Schutte, J.A. Reinbolt, B.J. Fregly, R.T. Haftka,
A.D. George, Parallel global optimization with the particle swarm
algorithm. International journal for numerical methods in engineering
\textbf{61}, pp. 2296-2315, 2004.

\bibitem{go_par3}Regis, R. G., \& Shoemaker, C. A. (2009). Parallel
stochastic global optimization using radial basis functions. INFORMS
Journal on Computing, 21(3), 411-426.

\bibitem{pga_survey1}R. Shonkwiler, Parallel genetic algorithms.
In ICGA (pp. 199-205), 1993.

\bibitem{pga_survey2}E. Cantú-Paz, A survey of parallel genetic algorithms,
Calculateurs paralleles, reseaux et systems repartis \textbf{10},
pp. 141-171, 1998.

\bibitem{Marini} Marini, F. \& Walczak, B. (2015). Particle swarm
optimization (PSO). A tutorial. Chemometrics and Intelligent Laboratory
Systems. Volume 149, Part B, p.:153-165. Doi: https://doi.org/10.1016/j.chemolab.2015.08.020

\bibitem{Garc=0000EDa-Gonzalo} García-Gonzalo, E. \& Fernández-Martínez,
J. L. (2012). A Brief Historical Review of Particle Swarm Optimization
(PSO). Journal of Bioinformatics and Intelligent Control, Volume 1,
Number 1, June 2012, pp. 3-16(14). American Scientific Publishers.
DOI:https://doi.org/10.1166/jbic.2012.1002 

\bibitem{Jain} Jain, M., Saihjpal, V., Singh, N. \& Singh, S.B. (2022).
An Overview of Variants and Advancements of PSO Algorithm. MDPI, Applied
Sciences: 12, 8392. Doi:https://doi.org/10.3390/ app12178392 . 

\bibitem{psophysics1}Anderson Alvarenga de Moura Meneses, Marcelo
Dornellas, Machado Roberto Schirru, Particle Swarm Optimization applied
to the nuclear reload problem of a Pressurized Water Reactor, Progress
in Nuclear Energy \textbf{51}, pp. 319-326, 2009.

\bibitem{psophysics2}Ranjit Shaw, Shalivahan Srivastava, Particle
swarm optimization: A new tool to invert geophysical data, Geophysics
\textbf{72}, 2007.

\bibitem{psochem1}C. O. Ourique, E.C. Biscaia, J.C. Pinto, The use
of particle swarm optimization for dynamical analysis in chemical
processes, Computers \& Chemical Engineering \textbf{26}, pp. 1783-1793,
2002.

\bibitem{psochem2}H. Fang, J. Zhou, Z. Wang et al, Hybrid method
integrating machine learning and particle swarm optimization for smart
chemical process operations, Front. Chem. Sci. Eng. \textbf{16}, pp.
274--287, 2022.

\bibitem{psomed1}M.P. Wachowiak, R. Smolikova, Yufeng Zheng, J.M.
Zurada, A.S. Elmaghraby, An approach to multimodal biomedical image
registration utilizing particle swarm optimization, IEEE Transactions
on Evolutionary Computation \textbf{8}, pp. 289-301, 2004.

\bibitem{psomed2}Yannis Marinakis. Magdalene Marinaki, Georgios Dounias,
Particle swarm optimization for pap-smear diagnosis, Expert Systems
with Applications \textbf{35}, pp. 1645-1656, 2008. 

\bibitem{psoecon}Jong-Bae Park, Yun-Won Jeong, Joong-Rin Shin, Kwang
Y. Lee, An Improved Particle Swarm Optimization for Nonconvex Economic
Dispatch Problems, IEEE Transactions on Power Systems \textbf{25},
pp. 156-16\textbf{216}6, 2010.

\bibitem{Feoktistov} Feoktistov, V. (2006). Differential Evolution.
In Search of Solutions. Optimization and Its Appl;ications, Springer.
Doi: https://doi.org/10.1007/978-0-387-36896-2 

\bibitem{Bilal} Bilal, M.P., Zaheer, H., Garcia-Hernandez, L. \&
Abraham, A. (2020). Differential Evolution: A review of more than
two decades of research. Engineering Applications of Artificial Intelligence
90 (2020) 103479. Doi: https://doi.org/10.1016/j.engappai.2020.103479 

\bibitem{de_app1}P. Rocca, G. Oliveri, A. Massa, Differential Evolution
as Applied to Electromagnetics, IEEE Antennas and Propagation Magazine.
\textbf{53}, pp. 38-49, 2011.

\bibitem{de_app2}W.S. Lee, Y.T. Chen, Y. Kao, Optimal chiller loading
by differential evolution algorithm for reducing energy consumption,
Energy and Buildings \textbf{43}, pp. 599-604, 2011.

\bibitem{de_app3}Y. Yuan, H. Xu, Flexible job shop scheduling using
hybrid differential evolution algorithms, Computers \& Industrial
Engineering \textbf{65}, pp. 246-260, 2013.

\bibitem{de_app4}L. Xu, H. Jia, C. Lang, X. Peng, K. Sun, A Novel
Method for Multilevel Color Image Segmentation Based on Dragonfly
Algorithm and Differential Evolution, IEEE Access \textbf{7}, pp.
19502-19538, 2019.

\bibitem{Marti} Marti, R., Resende, M.G.C. \& Ribeiro, C. (2013).
Multi-start methods for combinatorial optimization. European Journal
of Operational Research Volume 226, Issue 1, 1 April 2013, Pages 1-8.
Doi: https://doi.org/10.1016/j.ejor.2012.10.012 

\bibitem{Marti Moreno} Marti, R., Moreno-Vega, J. \& Duarte, A. (2010).
Advanced Multi-start Methods. Handbook of Metaheuristics, pp: 265--281.

\bibitem{Tu Mayne} Tu, W. \& Mayne, R.W. (2002). Studies of multi-start
clustering for global optimization. International Journal for Numerical
Methods in Engineering. Doi: https://doi.org/10.1002/nme.400. 

\bibitem{bfgs_main}Dai, Y. H. (2002). Convergence properties of the
BFGS algoritm. SIAM Journal on Optimization, 13(3), 693-701.

\bibitem{Tsoulos} Tsoulos, I.G. Modifications of real code genetic
algorithm for global optimization. Appl. Math. Comput. 2008, 203,
598--607.

\bibitem{Gaviano} M. Gaviano, D.E. Ksasov, D. Lera, Y.D. Sergeyev,
Software for generation of classes of test functions with known local
and global minima for global optimization, ACM Trans. Math. Softw.
\textbf{29}, pp. 469-480, 2003.

\bibitem{Lennard} J.E. Lennard-Jones, On the Determination of Molecular
Fields, Proc. R. Soc. Lond. A \textbf{ 106}, pp. 463--477, 1924.

\bibitem{Zabinsky} Z.B. Zabinsky, D.L. Graesser, M.E. Tuttle, G.I.
Kim, Global optimization of composite laminates using improving hit
and run, In: Recent advances in global optimization, pp. 343-368,
1992.

\bibitem{pde}Charilogis, V.; Tsoulos, I.G. A Parallel Implementation
of the Differential Evolution Method. Analytics 2023, 2, 17--30.

\bibitem{ppso}Charilogis, V.; Tsoulos, I.G.; Tzallas, A. (2023).
An Improved Parallel Particle Swarm Optimization. SN Computer Science
(2023) 4:766

\bibitem{Floudas} C.A. Floudas, P.M. Pardalos, C. Adjiman, W. Esposoto,
Z. G$\ddot{\mbox{u}}$m$\ddot{\mbox{u}}$s, S. Harding, J. Klepeis,
C. Meyer, C. Schweiger, Handbook of Test Problems in Local and Global
Optimization, Kluwer Academic Publishers, Dordrecht, 1999.

\bibitem{Montaz Ali}M. Montaz Ali, Charoenchai Khompatraporn, Zelda
B. Zabinsky, A Numerical Evaluation of Several Stochastic Algorithms
on Selected Continuous Global Optimization Test Problems, Journal
of Global Optimization \textbf{31}, pp 635-672, 2005.

\bibitem{Gao}Gao, Z.M., Zhao, J., Hu,Y.R., Chen, H.F. (2021).The
Challenge for the Nature - Inspired Global Optimization Algorithms:
Non-Symmetric Benchmark Functions. IEEE Access, July 26, 2021.

\bibitem{openMP} R. Chandra, L. Dagum, D. Kohr, D. Maydan,J. McDonald
and R. Menon, Parallel Programming in OpenMP, Morgan Kaufmann Publishers
Inc., 2001.

\end{thebibliography}

\end{adjustwidth}{}
\end{document}
