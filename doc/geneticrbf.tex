%% LyX 2.3.6 created this file.  For more info, see http://www.lyx.org/.
%% Do not edit unless you really know what you are doing.
\documentclass[english]{article}
\usepackage[T1]{fontenc}
\usepackage[latin9]{inputenc}
\usepackage{float}
\usepackage{url}
\usepackage{amstext}

\makeatletter

%%%%%%%%%%%%%%%%%%%%%%%%%%%%%% LyX specific LaTeX commands.
%% Because html converters don't know tabularnewline
\providecommand{\tabularnewline}{\\}
\floatstyle{ruled}
\newfloat{algorithm}{tbp}{loa}
\providecommand{\algorithmname}{Algorithm}
\floatname{algorithm}{\protect\algorithmname}

\makeatother

\usepackage{babel}
\begin{document}
\title{A two phase evolutionary method to train RBF networks}
\author{Ioannis G. Tsoulos, Alexandros Tzallas, Evangelos Karvounis}
\date{Department of Informatics and Telecommunications, University of Ioannina,
Greece}
\maketitle
\begin{abstract}
This article proposes a two phase hybrid method to train RBF neural
networks for classification and regression problems. During the first
phase a range for the critical parameters of the RBF network is estimated
and in the second phase a genetic algorithm is incorporated to locate
the best RBF neural network for the underlying problem. The method
is compared against the traditional training method of RBF neural
networks on a wide series of classification and regression problems
from the relevant literature and the results are reported.
\end{abstract}

\section{Introduction}

In machine learning appear many practical problems such as classification
and regression problems. A good programming tool that can be used
to tackle this problem is Radial Basis Function (RBF) networks\cite{rbf1}.
These networks typically are expressed as a function:\textbf{
\begin{equation}
y(x)=\sum_{i=1}^{k}w_{i}\phi\left(\left\Vert x-c_{i}\right\Vert \right)\label{eq:firstrbf}
\end{equation}
}where $x$ is the input pattern, the vector $\overrightarrow{w}$
is called the weight vector and $y(x)$ is the predicted value of
the network. RBF networks are feedforward neural networks\cite{ann-bishop}
with three computational layers: 
\begin{enumerate}
\item The input layer, where the problem is presented in the form of patterns
to the neural network
\item The processing layer, where a computation is performed using the Gaussian
processing units $\phi(x)$. These units can have many forms in the
relevant literature but the most used form is the Gaussian function
expressed as: \textbf{
\begin{equation}
\phi(x)=\exp\left(-\frac{\left(x-c\right)^{2}}{\sigma^{2}}\right)
\end{equation}
}The value $\phi(x)$ depends only on the distance of vector $x$
from some other vector $c$, which typically is called centroid. 
\item The output layer where the output of every function $\phi(x)$ is
multiplied by a corresponding weight value $w_{i}$.
\end{enumerate}
RBF networks have been used in many classification and regression
problems from the areas of physics \cite{rbfphysics1,rbfphysics2,rbfphysics3,rbfphysics4},
medicine \cite{rbfmed1,rbfmed2,rbfmed3},\textbf{ }solution of differential
equations\textbf{ }\cite{rbfde1,rbfde2}, chemistry \cite{rbfchemistry1,rbfchemistry2,rbfchemistry3},
economics \cite{rbfecon1,rbfecon2,rbfecon3}, digital communications
\cite{rbfnetwork1,rbfnetwork2}\textbf{ }etc. Because of the extensive
use of RBF networks, many methods have been proposed in the recent
literature to enhance them. There are methods that parallelize the
RBF networks \cite{rbfparallel,rbfgpu}, methods that improve the
initialization of the RBF parameters\textbf{ }\cite{rbfinit1,rbfinit2,rbfinit3},
methods that alter the architecture of the network\textbf{ }\cite{rbfprun1,rbfprun2,rbfprun3},
methods aimed to locate the best set of the RBF parameters with global
optimization techniques\cite{rbfpso1,rbfpso2,rbfdiff1} etc. This
article transforms the problem of RBF training into an optimization
problem and applies a modified genetic algorithm technique to solve
it. The global optimization problem is defined as :
\begin{equation}
\min\left(E\left(y\right)\right)=\sum_{i=}^{m}\left(y\left(x_{i}\right)-t_{i}\right)^{2}\label{eq:eqrbf}
\end{equation}
where $m$ is the total number of input patterns and $t_{i}$ is the
output for pattern $x_{i}$. The suggested approach has two phases:
firstly reasonable bounds for the RBF parameters are estimated using
the Kmeans \cite{kmeans} algorithm and in the second phase the modified
algorithm is used to solve the problem of equation \ref{eq:eqrbf}
inside the bounds located in the first phase.\textbf{ }

The rest of this paper is organized as follows: in section \ref{sec:Method-description}
the proposed method is described, in section \ref{sec:Experiments}
the conducted experiments are listed and the proposed method is compared
against the traditional training of RBF networks and finally in section
\ref{sec:Conclusions} some conclusions are derived.

\section{Method description\label{sec:Method-description}}

The proposed method can be divided into two main phases: during the
first phase an approximation for the bound of RBF parameters is made
using the K-Means algorithm in the second phase the optimization problem
is solved using a modified genetic algorithm. These phases are outlined
in detail in the following subsections.

\subsection{Bound location phase \label{subsec:Bound-location-phase}}

The proposed genetic algorithm has chromosomes with dimension $(d+1)\times k$,
where $d$ is the dimension of the input problem, i.e. the dimension
of the vector $\overrightarrow{x_{i}}$ in equation \ref{eq:eqrbf}
and $k$ is the total number of processing units of the RBF network.
The layout of each chromosome is presented in Figure \ref{fig:The-layout-of}.
Every center $\overrightarrow{c_{i}}$ in the equation \ref{eq:firstrbf}
is a vector of dimension $d$ and also an additional parameter is
reserved for the parameter $\sigma$ of every $\phi(x)$ function.
The centroids and the corresponding variances are estimated using
the Kmeans algorithm that described in algorithm \ref{alg:The-KMeans-algorithm.}.
The value $\sigma_{i}$ for every $\phi_{i}(x)$ is calculated as:
\begin{equation}
\sigma_{i}=\sum_{j=1}^{d}s_{ij}^{2}
\end{equation}
After the estimation of $c_{i}$ and $\sigma_{i}$ the vectors $\overrightarrow{L},\ \overrightarrow{R}$
with dimension $(d+1)\times k$ are constructed. These vectors will
serve as the bounds for the chromosomes of the genetic population.
These vector are constructed using the following procedure:
\begin{enumerate}
\item \textbf{Set} m=0
\item \textbf{Set} $F>1$ 
\item \textbf{For} $i=1..k$ \textbf{do}
\begin{enumerate}
\item \textbf{For} $j=1..d$ \textbf{do}
\begin{enumerate}
\item \textbf{Set} $L_{m}$=$-F\times c_{ij}$, $R_{m}$=$F\times c_{ij}$
\item \textbf{Set} $m=m+1$
\end{enumerate}
\item \textbf{EndFor}
\item \textbf{Set} $L_{m}=-F\times\sigma_{i}$, $R_{m}=F\times\sigma_{i}$
\item \textbf{Set} $m=m+1$
\end{enumerate}
\item \textbf{EndFor}
\end{enumerate}
\begin{figure}

\caption{The layout of the chromosomes in the proposed genetic algorithm.\label{fig:The-layout-of}}

\centering{}%
\begin{tabular}{|c|c|c|c|c|c|c|c|c|c|c|c|c|c|c|c|}
\hline 
$c_{11}$ & $c_{12}$ & ... & $c_{1d}$ & $\sigma_{1}$ & $c_{21}$ & $c_{22}$ & ... & $c_{2d}$ & $\sigma_{2}$ & ... & $c_{k1}$ & $c_{k2}$ & ... & $c_{kd}$ & $\sigma_{k}$\tabularnewline
\hline 
\end{tabular}
\end{figure}
\begin{algorithm}
\caption{The KMeans algorithm.\label{alg:The-KMeans-algorithm.}}

\begin{enumerate}
\item \textbf{Repeat}
\begin{enumerate}
\item $S_{j}=\left\{ \right\} ,\ j=1..k$
\item \textbf{For} every sample $x_{i}$ \textbf{Do}
\begin{enumerate}
\item \textbf{Set} $j^{*}=\min_{i=1}^{k}\left\{ D\left(x_{i},c_{j}\right)\right\} $,
where $j^{*}$ is the nearest center for sample $x_{i}$. 
\item \textbf{Set} $S_{j^{*}}=S_{j^{*}}\cup\left\{ x_{i}\right\} $.
\end{enumerate}
\item \textbf{EndFor}
\item \textbf{For} every center $c_{j}$ \textbf{Do}
\begin{enumerate}
\item \textbf{Set} $M_{j}$=number of elements in $S_{j}$
\item \textbf{Update} $c_{j}$
\[
c_{j}=\frac{1}{M_{j}}\sum_{i=1}^{M_{j}}x_{i}
\]
\end{enumerate}
\item \textbf{EndFor} 
\end{enumerate}
\item \textbf{Calculate} the corresponding variances 
\[
s_{j}^{2}=\frac{\sum_{i=1}^{M_{j}}\left(x_{i}-c_{j}\right)^{2}}{M_{j}}
\]
\item \textbf{Terminate} when $c_{j}$ no longer change.
\end{enumerate}
\end{algorithm}


\subsection{Main algorithm }

The basic steps of the main algorithm are given below:
\begin{enumerate}
\item \textbf{Initialization Step}
\begin{enumerate}
\item \textbf{Read} the train set with $m$ of $d$ dimension. 
\item \textbf{Set} $k$ the number of nodes for the RBF network.
\item \textbf{Estimate} the vectors $\overrightarrow{L},\ \overrightarrow{R}$
using the procedure of subsection \ref{subsec:Bound-location-phase}. 
\item \textbf{Initialize} a genetic population of $N_{C}$ random chromosomes
inside $[L,R]$. 
\item \textbf{Set} the selection rate $P_{s}\in[0,1]$, the mutation rate
$P_{M}\in[0,1]$ , $iter=0,$ and $i_{max}$ the maximum number of
generations.
\end{enumerate}
\item \textbf{Evaluation Step}

\textbf{For} every chromosome $g$ \textbf{calculate} using the procedure
defined in subsection \ref{subsec:Fitness-evaluation} the fitness
$f_{g}$.
\item \textbf{Genetic step}
\begin{enumerate}
\item \textbf{Select} $P_{s}\times N_{c}$ parents from the population using
tournament selection, i.e. create subgroups of $T>2$ chromosomes
and select the one with the best fitness value as the parent. 
\item \textbf{Crossover}: For every pair $(x,y)$ of selected parents create
two new offsprings $\tilde{x}$ and $\tilde{y}$:
\begin{eqnarray}
\tilde{x_{i}} & = & a_{i}x_{i}+\left(1-a_{i}\right)y_{i}\nonumber \\
\tilde{y_{i}} & = & a_{i}y_{i}+\left(1-a_{i}\right)x_{i}\label{eq:crossover_ali-1}
\end{eqnarray}
with $a_{i}$ a random number and $a_{i}\in[-0.5,1.5]$ \cite{kaeloali}.
\item \textbf{Mutation}: For every element of each chromosome create a random
number $r\in\left[0,1\right]$. If $r\le P_{m}$ then change randomly
this element.
\item \textbf{Replace} the $P_{s}\times N_{c}$ worst chromosomes in the
population with the generated offsprings. 
\end{enumerate}
\item \textbf{Termination Check Step}
\begin{enumerate}
\item \textbf{Set} $iter=iter+1$ 
\item \textbf{Terminate} if the termination criteria of subsection \ref{subsec:Stopping-rule}
are satisfied, \textbf{else Goto} Evaluation Step.
\end{enumerate}
\end{enumerate}

\subsection{Fitness evaluation\label{subsec:Fitness-evaluation}}

In this step a valid RBF network $y(x)=\sum_{i=1}^{k}w_{i}\phi\left(\left\Vert x-c_{i}\right\Vert \right)$,
is created using the chromosome $g$ and subsequently is trained using
the typical training procedure for RBF networks. The main steps to
calculate the fitness $f_{g}$ of a chromosome $g$ are the following:
\begin{enumerate}
\item \textbf{Decode} the chromosome $g$ to the parts (centers and variances)
of the RBF network as defined by the layout of Figure \ref{fig:The-layout-of}.
\item \textbf{Calculate} the output vectors\textbf{ $w_{1},w_{w},\ldots,w_{k}$
}by solving the an induced system of equations:
\begin{enumerate}
\item \textbf{Set} $W=w_{kj}$ the matrix of $k$ weights, $\Phi=\phi_{j}\left(x_{i}\right)$
and $T=\left\{ t_{i}\right\} $. 
\item \textbf{Solve: 
\begin{equation}
\Phi^{T}\left(T-\Phi W^{T}\right)=0
\end{equation}
}givivg\textbf{:
\begin{equation}
W^{T}=\left(\Phi^{T}\Phi\right)^{-1}\Phi^{T}T=\Phi^{\dagger}T\label{eq:eqoutput}
\end{equation}
}The matrix $\Phi^{\dagger}=\left(\Phi^{T}\Phi\right)^{-1}\Phi^{T}$
is the pseudo-inverse of $\Phi$, with\textbf{
\begin{equation}
\Phi^{\dagger}\Phi=I
\end{equation}
}
\end{enumerate}
\item \textbf{Set} $f_{g}=$$\sum_{i=}^{m}\left(y\left(x_{i}\right)-t_{i}\right)^{2}$
\end{enumerate}

\subsection{Stopping rule\label{subsec:Stopping-rule}}

Define as $g_{best}$ the best chromosome in the population and define
as $\sigma^{(\mbox{iter})}$ the variance of best fitness $f\left(g_{\mbox{best}}\right)$
at generation iter. If fitness $f\left(g_{\mbox{best}}\right)$ has
not improved for a number of generations, then probably the algorithm
should terminate. Hence, the termination rule is defined as:

\begin{equation}
\text{\mbox{iter}\ensuremath{\ge i_{\mbox{max}}}}\ \text{\mbox{OR}\ }\sigma^{(\mbox{iter})}\le\frac{\sigma^{(\mbox{klast})}}{2}\label{eq:termination_mine-1}
\end{equation}
where $\mbox{klast}$ is the last generation where a new minimum was
found.

\section{Experiments \label{sec:Experiments}}

In order to evaluate the performance of the proposed method, comparative
experiments were performed on a series of well - known classification
and regression datasets from the relevant literature.

\subsection{Experimental setup}

The RBF network was coded in ANSI C++, using the Armadillo library
\cite{Armadillo} and the optimization was performed using the Genetic
optimization method of the optimization package OPTIMUS, that is freely
available from \url{https://github.com/itsoulos/OPTIMUS/}. Also,
to have more reliability in the results the common used method of
10 - fold cross validation was used, which means that the the original
data was randomly partitioned into 10 equal sized subsamples. Subsequently,
10 independent experiments were conducted: in each experiments one
subsample is used as the testing data and all the others as the training
data. The average error on the test data is the total test error.
All the experiments were executed 30 times with different initialization
for the random generator each time. The random generator used was
the function drand48() of C programming language.\textbf{ }The execution
environment was an Intel Xeon E5-2630 multi core machine using the
OpenMP library \cite{openmp} for parallelization and the Ubuntu Linux
operating system. The parameters for the genetic algorithm are displayed
in Table \ref{tab:Experimental-parameters.}.

\begin{table}
\caption{Experimental parameters.\label{tab:Experimental-parameters.}}

\centering{}%
\begin{tabular}{|c|c|}
\hline 
PARAMETER & VALUE\tabularnewline
\hline 
\hline 
$k$ & 10\tabularnewline
\hline 
$N_{c}$ & 200\tabularnewline
\hline 
$P_{s}$ & 0.10\tabularnewline
\hline 
$P_{m}$ & 0.05\tabularnewline
\hline 
$F$ & 3.0\tabularnewline
\hline 
$i_{max}$ & 200\tabularnewline
\hline 
\end{tabular}
\end{table}


\subsection{Experimental datasets }

The classification problems used for the experiments were found in
most cased in two internet databases:
\begin{enumerate}
\item UCI dataset repository, \url{https://archive.ics.uci.edu/ml/index.php}
\item Keel repository, \url{https://sci2s.ugr.es/keel/datasets.php}\cite{Keel}.
\end{enumerate}
The following classification datasets were used:
\begin{enumerate}
\item \textbf{Alcohol} dataset, a dataset about Alcohol consumption \cite{Alcohol}.
\item \textbf{Appendictis} dataset, proposed in \cite{appendicitis}. 
\item \textbf{Australian} dataset, the dataset is related to credit card
applications.
\item \textbf{Balance} dataset, which models psychological experimental
results. 
\item \textbf{Cleveland} dataset. A dataset obtained from the V.A. Medical
Center, Long Beach and Cleveland Clinic Foundation aimed to detect
the presence of heart disease in patients.
\item \textbf{Dermatology} dataset, which is used for differential diagnosis
of erythemato-squamous diseases. 
\item \textbf{Glass} dataset. The dataset contains glass component analysis
for glass pieces that belong to 6 classes. 
\item \textbf{Hayes roth} dataset. This dataset\cite{hayesroth} contains
\textbf{5} numeric-valued attributes and 132 patterns. 
\item \textbf{Heart} dataset, which is used to identify the absence or presence
of heart disease. 
\item \textbf{HouseVotes} dataset. This data set includes votes for each
of the U.S. House of Representatives Congressmen on the 16 key votes. 
\item \textbf{Ionosphere} dataset. The ionosphere dataset contains data
from the Johns Hopkins Ionosphere database.
\item \textbf{Liverdisorder}: This dataset contains blood analysis data
from people with liver disorders. 
\item \textbf{Mammographic} dataset. This dataset be used to identify the
severity (benign or malignant) of a mammographic mass lesion from
BI-RADS attributes and the patient's age. It contains 830 patterns
of 5 features each.
\item \textbf{Parkinsons} dataset. This dataset is composed of a range of
biomedical voice measurements from 31 people, 23 with Parkinson's
disease (PD)\cite{parkinsons}.
\item \textbf{Pima} dataset. The Pima Indians Diabetes dataset contains
768 examples of 8 attributes each that are classified into two categories:
healthy and diabetic.
\item \textbf{Popfailures} dataset. This dataset contains records of simulation
crashes encountered during climate model uncertainty quantification
(UQ) ensembles. 
\item \textbf{Regions2} dataset. It is created from liver biopsy images
of patients with hepatitis C \cite{regions}. From each region in
the acquired images, 18 shape-based and color-based features were
extracted, while it was also annotated form medical experts. The resulting
dataset includes 600 samples belonging into 6 classes.
\item \textbf{Ring} dataset. It is an 20 dimensional problem with two classes.
Each class is drawn from a multivariate normal distribution.
\item \textbf{Saheart} dataset. The dataset is about to categorize persons
if have a coronary heart disease. The dataset contains 462 patterns
with 9 features each.
\item \textbf{Segment} dataset. This database contains patterns from a database
of 7 outdoor images (classes). The dataset contains 2310 patterns
with 19 features each.
\item \textbf{Sonar} dataset. This data set contains signals obtained from
a variety of different aspect angles, spanning 90 degrees for mines
and 180 degrees for rocks. Each pattern is a set of 60 numbers in
the range 0.0 to 1.0, where each number represents the energy within
a particular frequency band, integrated over a certain period of time.
The output attribute contains the letter R if the object is a rock
and M if it is a mine (metal cylinder).
\item \textbf{Spiral} dataset: The spiral artificial dataset contains 1000
two-dimensional examples that belong to two classes (500 examples
each). The number of the features is 2. The data in the first class
are created using the following formula: $x_{1}=0.5t\cos\left(0.08t\right),\ x_{2}=0.5t\cos\left(0.08t+\frac{\pi}{2}\right)$
and the second class data using\textbf{: $x_{1}=0.5t\cos\left(0.08t+\pi\right),\ x_{2}=0.5t\cos\left(0.08t+\frac{3\pi}{2}\right)$}
\item \textbf{Tae} dataset. The data consist of evaluations of teaching
performance over three regular semesters and two summer semesters
of 151 teaching assistant (TA) assignments at the Statistics Department
of the University of Wisconsin-Madison.
\item \textbf{Thyroid} dataset, which concerns Thyroid disease records\cite{thyroid}.
\item \textbf{Wdbc} dataset. The Wisconsin diagnostic breast cancer dataset
contains data for breast tumors. 
\item \textbf{Wine} dataset. The wine recognition dataset contains data
from wine chemical analysis.
\item \textbf{Eeg} datasets. As an real word example, consider an EEG dataset
described in \cite{key-20} is used here. The dataset consists of
five sets (denoted as Z, O, N, F and S) each containing 100 single-channel
EEG segments each having 23.6 sec duration. With different combinations
of these sets the produced datasets are Z\_F\_S, ZO\_NF\_S, ZONF\_S.
\item \textbf{Zoo} dataset. A simple database where the task is to classify
animals in seven predefined classes and most of the attributes are
boolean-valued. 
\end{enumerate}
The regression datasets are in most cases available from the Statlib
URL \url{ftp://lib.stat.cmu.edu/datasets/index.html }: 
\begin{enumerate}
\item \textbf{Abalone} dataset. This data set can be used to obtain a model
to predict the age of abalone from physical measurements. 
\item \textbf{Airfoil }dataset, which is used by the NASA for a series of
aerodynamic and acoustic tests \cite{airfoil}. 
\item \textbf{Anacalt} dataset. This contains information about the decisions
taken by a supreme court. 
\item \textbf{BK} dataset. This dataset comes from Smoothing Methods in
Statistics \cite{Stat} and is used to estimate the points scored
per minute in a basketball game. 
\item \textbf{BL} dataset: This dataset can be downloaded from StatLib.
It contains data from an experiment on the affects of machine adjustments
on the time to count bolts. 
\item \textbf{Concrete} dataset. This dataset is taken from civil engineering\cite{concrete}. 
\item \textbf{Housing} dataset. This dataset was taken from the StatLib
library which is maintained at Carnegie Mellon University and it is
described in \cite{key23}.
\item \textbf{Laser} dataset. Dataset used in laser experiments. 
\item \textbf{MB} dataset. This dataset is available from Smoothing Methods
in Statistics \cite{key21} and it includes 61 patterns. 
\item \textbf{NT} dataset. This dataset contains data from \cite{ntdataset}
that examined whether the true mean body temperature is 98.6 F. 
\item \textbf{Quake} dataset. The objective here is to approximate the strength
of a earthquake. 
\end{enumerate}

\subsection{Experimental results}

The results for the classification datasets are listed in Table \ref{tab:Experimental-results}
and for the regression datasets the results are reported in Table
\ref{tab:exp2}. For the first case the average classification error
is reported and for the case of regression datasets the total test
error is reported. The column KRBF denotes the classic RBF training
method and the column PROPOSED denotes the proposed method. The KBF
simply consists of two phases: in the first phase centers and variances
are estimated through the Kmeans algorithm and in the second phase
a system of equations is solved to obtain the weights $w_{i}$ of
the RBF network. 

From the experimental results it is clear that the proposed method
is significantly superior to traditional technique in almost all datasets.
In the proposed method the appropriate initialization interval is
found for the parameters of RBF using Kmeans. A parallel genetic algorithm
was then applied to this previous value range creating a variety of
neural networks. This combination of techniques obviously has very
good results as it combines a very efficient clustering method and
an excellent optimization method that is ideally parallelized. Of
course, the new method requires much more execution time, due to the
presence of the genetic algorithm but the parallel execution of the
software drastically reduces this time. 

\begin{table}
\caption{Experimental results for classification problems.\label{tab:Experimental-results}}

\centering{}%
\begin{tabular}{|c|c|c|}
\hline 
DATASET & KRBF & PROPOSED\tabularnewline
\hline 
\hline 
Alcohol & 46.63\% & 21.86\%\tabularnewline
\hline 
Appendicitis & 12.23\% & 16.03\%\tabularnewline
\hline 
Australian & 34.89\% & 22.97\%\tabularnewline
\hline 
Balance & 33.42\% & 12.88\%\tabularnewline
\hline 
Cleveland & 67.10\% & 51.75\%\tabularnewline
\hline 
Dermatology & 62.34\% & 37.37\%\tabularnewline
\hline 
Glass & 50.16\% & 49.16\%\tabularnewline
\hline 
Hayes Roth & 64.36\% & 35.26\%\tabularnewline
\hline 
Heart & 31.20\% & 17.80\%\tabularnewline
\hline 
HouseVotes & 6.13\% & 3.67\%\tabularnewline
\hline 
Ionosphere & 16.22\% & 10.33\%\tabularnewline
\hline 
Liverdisorder & 30.84\% & 28.73\%\tabularnewline
\hline 
Mammographic & 21.38\% & 17.25\%\tabularnewline
\hline 
Parkinsons & 17.42\% & 17.37\%\tabularnewline
\hline 
Pima & 25.78\% & 24.00\%\tabularnewline
\hline 
Popfailures & 7.04\% & 5.44\%\tabularnewline
\hline 
Regions2 & 38.29\% & 25.81\%\tabularnewline
\hline 
Ring & 21.65\% & 2.09\%\tabularnewline
\hline 
Saheart & 32.19\% & 29.38\%\tabularnewline
\hline 
Segment & 59.68\% & 39.44\%\tabularnewline
\hline 
Sonar & 27.85\% & 19.62\%\tabularnewline
\hline 
Spiral & 44.87\% & 18.98\%\tabularnewline
\hline 
Tae & 60.07\% & 52.44\%\tabularnewline
\hline 
Thyroid & 10.52\% & 7.12\%\tabularnewline
\hline 
Wdbc & 7.27\% & 5.29\%\tabularnewline
\hline 
Wine & 31.41\% & 8.67\%\tabularnewline
\hline 
Z\_F\_S & 13.16\% & 4.21\%\tabularnewline
\hline 
ZO\_NF\_S & 9.02\% & 4.17\%\tabularnewline
\hline 
ZONF\_S & 4.03\% & 2.18\%\tabularnewline
\hline 
ZOO & 21.93\% & 9.00\%\tabularnewline
\hline 
\end{tabular}
\end{table}
\begin{table}
\caption{Experiments for regression datasets.\label{tab:exp2}}

\centering{}%
\begin{tabular}{|c|c|c|}
\hline 
DATASET & KMEANS RBF & PROPOSED\tabularnewline
\hline 
\hline 
ABALONE & 2559.48 & 1960.22\tabularnewline
\hline 
AIRFOIL & 5.49 & 0.58\tabularnewline
\hline 
ANACALT & 11.628 & 0.003\tabularnewline
\hline 
BK & 0.17 & 0.23\tabularnewline
\hline 
BL & 0.05 & 0.0009\tabularnewline
\hline 
CONCRETE & 1.15 & 0.52\tabularnewline
\hline 
HOUSING & 2884.09 & 693.22\tabularnewline
\hline 
LASER & 2.35 & 1.04\tabularnewline
\hline 
MB & 11.33 & 0.63\tabularnewline
\hline 
NT & 72.14 & 0.09\tabularnewline
\hline 
QUAKE & 15.36 & 7.86\tabularnewline
\hline 
\end{tabular}
\end{table}


\section{Conclusions\label{sec:Conclusions}}

A two phase method was proposed in this article to train RBF neural
networks for classification and regression problems.\textbf{ }Firstly,
a common used clustering method was used to estimate an interval for
the critical parameters of the neural network. Subsequently, a parallel
genetic algorithm was incorporated to locate the best RBF network
with good generalization capabilities. The used software was coded
using ANSI C++ and open source libraries such as the Armadillo library
and the OpenMP library for parallelization. Future research may include
\begin{enumerate}
\item Use parallel methods for the Kmeans clustering phase of the method.
\item Dynamic selection of K in Kmeans algorithm. 
\item More advanced stopping rules for the genetic algorithm.
\item Replace the Genetic algorithm with other optimization methods such
as Particle Swarm Optimization, Ant Colony Optimization etc. 
\end{enumerate}

\subsection*{Acknowledgments}

The experiments of this research work was performed at the high performance
computing system established at Knowledge and Intelligent Computing
Laboratory, Dept of Informatics and Telecommunications, University
of Ioannina, acquired with the project \textquotedbl Educational
Laboratory equipment of TEI of Epirus\textquotedbl{} with MIS 5007094
funded by the Operational Programme \textquotedbl Epirus\textquotedbl{}
2014-2020, by ERDF and national finds.

\subsection*{Compliance with Ethical Standards }

All authors declare that they have no conflict of interest. 
\begin{thebibliography}{10}
\bibitem{rbf1}J. Park and I. W. Sandberg, Universal Approximation
Using Radial-Basis-Function Networks, Neural Computation 3, pp. 246-257,
1991.

\bibitem{ann-bishop}C.M. Bishop, Neural networks and their applications,
Review of Scientific Instruments \textbf{65}, pp. 1803-1832,1994.

\bibitem{rbfphysics1}P. Teng, Machine-learning quantum mechanics:
Solving quantum mechanics problems using radial basis function networks,
Phys. Rev. E \textbf{98}, 033305, 2018.

\bibitem{rbfphysics2}R. Jovanovi\'{c}, A. Sretenovic, Ensemble of
radial basis neural networks with K-means clustering for heating energy
consumption prediction, FME Transactions \textbf{45}, pp. 51-57, 2017.

\bibitem{rbfphysics3}A. Alexandridis, E. Chondrodima, E. Efthimiou,
G. Papadakis, F. Vallianatos and D. Triantis, Large Earthquake Occurrence
Estimation Based on Radial Basis Function Neural Networks,IEEE Transactions
on Geoscience and Remote Sensing \textbf{52}, pp. 5443-5453, 2014.

\bibitem{rbfphysics4}V.I. Gorbachenko, M.V. Zhukov, Solving boundary
value problems of mathematical physics using radial basis function
networks, Comput. Math. and Math. Phys. \textbf{57}, pp. 145--155,
2017. 

\bibitem{rbfmed1}Y.P. Wang, J.W. Dang, Q. Li and S. Li, Multimodal
medical image fusion using fuzzy radial basis function neural networks,
in 2007 International Conference on Wavelet Analysis and Pattern Recognition,
pp. 778-782, 2017.

\bibitem{rbfmed2}S, Mehrabi, M. Maghsoudloo, H. Arabalibeik, R. Noormand
and Y. Nozari, Congestive heart failure, Chronic obstructive pulmonary
disease, Clinical decision support system, Multilayer perceptron neural
network and radial basis function neural network, Expert Systems with
Applications \textbf{36}, pp. 6956-6959, 2009.

\bibitem{rbfmed3}M. Veezhinathan, S. Ramakrishnan, Detection of Obstructive
Respiratory Abnormality Using Flow--Volume Spirometry and Radial
Basis Function Neural Networks, J Med Syst \textbf{31}, pp. 461, 2007.

\bibitem{rbfde1}Nam Mai-Duy, Thanh Tran-Cong, Numerical solution
of differential equations using multiquadric radial basis function
networks, Neural Networks 14, pp. 185-199, 2001.

\bibitem{rbfde2}N. Mai-Duy, Solving high order ordinary differential
equations with radial basis function networks. Int. J. Numer. Meth.
Engng. \textbf{62}, pp. 824-852, 2005.

\bibitem{rbfchemistry1}Chuanhao Wan and Peter de B. Harrington, Self-Configuring
Radial Basis Function Neural Networks for Chemical Pattern Recognition,
J. Chem. Inf. Comput. Sci. \textbf{39}, 1049--1056, 1999.

\bibitem{rbfchemistry2}Xiaojun Yao, Xiaoyun Zhang, Ruisheng Zhang,
Mancang Liu, Zhide Hu, Botao Fan, Prediction of enthalpy of alkanes
by the use of radial basis function neural networks, Computers \&
Chemistry \textbf{25}, pp. 475-482, 2001.

\bibitem{rbfchemistry3}A. Shahsavand, A. Ahmadpour, Application of
optimal RBF neural networks for optimization and characterization
of porous materials, Computers \& Chemical Engineering 29, pp. 2134-2143,
2005.

\bibitem{rbfecon1}J. A. Momoh and S. S. Reddy, Combined Economic
and Emission Dispatch using Radial Basis Function, 2014 IEEE PES General
Meeting | Conference \& Exposition, National Harbor, MD, 2014, pp.
1-5, doi: 10.1109/PESGM.2014.6939506.

\bibitem{rbfecon2}Jau-Jia Guo and P. B. Luh, Selecting input factors
for clusters of Gaussian radial basis function networks to improve
market clearing price prediction, IEEE Transactions on Power Systems
\textbf{18}, pp. 665-672, 2003.

\bibitem{rbfecon3}L. Falat, Z. Stanikova, M. Durisova, B. Holkova,
T. Potkanova, Application of Neural Network Models in Modelling Economic
Time Series with Non-constant Volatility, Procedia Economics and Finance
\textbf{34}, pp. 600-607, 2015.

\bibitem{rbfnetwork1}C. Laoudias, P. Kemppi and C. G. Panayiotou,
Localization Using Radial Basis Function Networks and Signal Strength
Fingerprints in WLAN, GLOBECOM 2009 - 2009 IEEE Global Telecommunications
Conference, Honolulu, HI, 2009, pp. 1-6, 2009.

\bibitem{rbfnetwork2}S. Chen, B. Mulgrew and P. M. Grant, A clustering
technique for digital communications channel equalization using radial
basis function networks, IEEE Transactions on Neural Networks \textbf{4},
pp. 570-590, 1993.

\bibitem{rbfparallel}M.G. Arenas, E. Parras-Guti�rrez, V.M. Rivas,
P.A. Castillo, M.J. Del Jesus, J.J. Merelo. Parallelizing the Design
of Radial Basis Function Neural Networks by Means of Evolutionary
Meta-algorithms. In: Cabestany J., Sandoval F., Prieto A., Corchado
J.M. (eds) Bio-Inspired Systems: Computational and Ambient Intelligence.
IWANN 2009. Lecture Notes in Computer Science, vol 5517. Springer,
Berlin, Heidelberg, 2009.

\bibitem{rbfgpu}A. Brandstetter and A. Artusi, \textquotedbl Radial
Basis Function Networks GPU-Based Implementation,\textquotedbl{} in
IEEE Transactions on Neural Networks \textbf{19}, pp. 2150-2154, 2008.

\bibitem{rbfinit1}L.I. Kuncheva, Initializing of an RBF network by
a genetic algorithm, Neurocomputing \textbf{14}, pp. 273-288, 1997.

\bibitem{rbfinit2}M. Kubat, Decision trees can initialize radial-basis
function networks, IEEE Transactions on Neural Networks. \textbf{9},
pp. 813-821, 1998.

\bibitem{rbfinit3}D. G. B. Franco and M. T. A. Steiner, New Strategies
for Initialization and Training of Radial Basis Function Neural Networks,
IEEE Latin America Transactions.\textbf{ 15}, pp. 1182-1188, 2017.

\bibitem{rbfprun1}E. Ricci, R. Perfetti, Improved pruning strategy
for radial basis function networks with dynamic decay adjustment,
Neurocomputing \textbf{69}, pp. 1728-1732, 2006.

\bibitem{rbfprun2}M. Bortman and M. Aladjem, A Growing and Pruning
Method for Radial Basis Function Networks, IEEE Transactions on Neural
Networks \textbf{20}, pp. 1039-1045, 2009.

\bibitem{rbfprun3}Guang-Bin Huang, P. Saratchandran and N. Sundararajan,
A generalized growing and pruning RBF (GGAP-RBF) neural network for
function approximation, IEEE Transactions on Neural Networks \textbf{16},
pp. 57-67, 2005.

\bibitem{rbfpso1}J. Y. Chen, Z. Qin and J. Jia, A PSO-Based Subtractive
Clustering Technique for Designing RBF Neural Networks, 2008 IEEE
Congress on Evolutionary Computation (IEEE World Congress on Computational
Intelligence), Hong Kong, pp. 2047-2052, 2008.

\bibitem{rbfpso2}A. Esmaeili and N. Mozayani, Adjusting the parameters
of radial basis function networks using Particle Swarm Optimization,
2009 IEEE International Conference on Computational Intelligence for
Measurement Systems and Applications, Hong Kong, pp. 179-181, 2009.

\bibitem{rbfdiff1}B. O'Hora, J. Perera and A. Brabazon, Designing
Radial Basis Function Networks for Classification Using Differential
Evolution, The 2006 IEEE International Joint Conference on Neural
Network Proceedings, Vancouver, BC, pp. 2932-2937, 2006.

\bibitem{Tsoulos1}I.G. Tsoulos, Modifications of real code genetic
algorithm for global optimization, Applied Mathematics and Computation
\textbf{203}, pp. 598-607, 2008.

\bibitem{kmeans}MacQueen, J.: Some methods for classification and
analysis of multivariate observations, in: Proceedings of the fifth
Berkeley symposium on mathematical statistics and probability, Vol.
1, No. 14, pp. 281-297, 1967. 

\bibitem{kaeloali}P. Kaelo, M.M. Ali, Integrated crossover rules
in real coded genetic algorithms, European Journal of Operational
Research \textbf{176}, pp. 60-76, 2007.

\bibitem{Armadillo}C. Sanderson, R. Curtin, Armadillo: a template-based
C++ library for linear algebra, Journal of Open Source Software \textbf{1},
pp. 26, 2016. 

\bibitem{Keel}J. Alcal�-Fdez, A. Fernandez, J. Luengo, J. Derrac,
S. Garc�a, L. S�nchez, F. Herrera. KEEL Data-Mining Software Tool:
Data Set Repository, Integration of Algorithms and Experimental Analysis
Framework. Journal of Multiple-Valued Logic and Soft Computing 17,
pp. 255-287, 2011.

\bibitem{openmp}R. Chandra, L. Dagum, D. Kohr, D. Maydan,J. McDonald
and R. Menon, Parallel Programming in OpenMP, Morgan Kaufmann Publishers
Inc., 2001.

\bibitem{Alcohol}Tzimourta, Katerina D. and Tsoulos, Ioannis and
Bilero, Thanasis and Tzallas, Alexandros T. and Tsipouras, Markos
G. and Giannakeas, Nikolaos, Direct Assessment of Alcohol Consumption
in Mental State Using Brain Computer Interfaces and Grammatical Evolution,
Inventions \textbf{3}, pp. 1-12, 2018.

\bibitem{appendicitis}Weiss, Sholom M. and Kulikowski, Casimir A.,
Computer Systems That Learn: Classification and Prediction Methods
from Statistics, Neural Nets, Machine Learning, and Expert Systems,
Morgan Kaufmann Publishers Inc, 1991.

\bibitem{hayesroth}B. Hayes-Roth, B., F. Hayes-Roth. Concept learning
and the recognition and classification of exemplars. Journal of Verbal
Learning and Verbal Behavior \textbf{16}, pp. 321-338, 1977.

\bibitem{parkinsons}Little MA, McSharry PE, Hunter EJ, Spielman J,
Ramig LO. Suitability of dysphonia measurements for telemonitoring
of Parkinson's disease. IEEE Trans Biomed Eng. 2009;56(4):1015. doi:10.1109/TBME.2008.2005954

\bibitem{regions}Giannakeas, N., Tsipouras, M.G., Tzallas, A.T.,
Kyriakidi, K., Tsianou, Z.E., Manousou, P., Hall, A., Karvounis, E.C.,
Tsianos, V., Tsianos, E. A clustering based method for collagen proportional
area extraction in liver biopsy images (2015) Proceedings of the Annual
International Conference of the IEEE Engineering in Medicine and Biology
Society, EMBS, 2015-November, art. no. 7319047, pp. 3097-3100. 

\bibitem{key-20}R.G. Andrzejak, K. Lehnertz, F. Mormann, C. Rieke,
P. David, and C. E. Elger, Indications of nonlinear deterministic
and finite-dimensional structures in time series of brain electrical
activity: Dependence on recording region and brain state, Phys. Rev.
E \textbf{64}, pp. 1-8, 2001.

\bibitem{airfoil}T.F. Brooks, D.S. Pope, and A.M. Marcolini. Airfoil
self-noise and prediction. Technical report, NASA RP-1218, July 1989. 

\bibitem{concrete}I.Cheng Yeh, Modeling of strength of high performance
concrete using artificial neural networks, Cement and Concrete Research.
\textbf{28}, pp. 1797-1808, 1998. 

\bibitem{key23}D. Harrison and D.L. Rubinfeld, Hedonic prices and
the demand for clean ai, J. Environ. Economics \& Management \textbf{5},
pp. 81-102, 1978.

\bibitem{key21}J.S. Simonoff, Smooting Methods in Statistics, Springer
- Verlag, 1996.

\bibitem{thyroid}Quinlan,J.R., Compton,P.J., Horn,K.A., and Lazurus,L.
(1986). Inductive knowledge acquisition: A case study. In Proceedings
of the Second Australian Conference on Applications of Expert Systems.
Sydney, Australia.

\bibitem{ntdataset}Mackowiak, P.A., Wasserman, S.S., Levine, M.M.,
1992. A critical appraisal of 98.6 degrees f, the upper limit of the
normal body temperature, and other legacies of Carl Reinhold August
Wunderlich. J. Amer. Med. Assoc. 268, 1578--1580

\bibitem{Stat}J.S. Simonoff, Smooting Methods in Statistics, Springer
- Verlag, 1996.
\end{thebibliography}

\end{document}
